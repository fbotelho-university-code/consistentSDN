%
% Modelo para relat￳rio da disciplina de Projecto de Engenharia Informatica
% do MEI.
%
% Incorpora elementos impostos pelo Regulamento de Estudos Pos-Graduados da
% Universidade de Lisboa (Deliberacao 1506/2006 - Diario da Repblica, 2.a s←rie 
% - n.o 209 - 30 de Outubro de 2006)
%
\documentclass[12pt,openright,twoside]{report}
\usepackage[show]{chato-notes}
\usepackage[utf8]{inputenc}
% Quem tiver problemas com os acentos, trocar utf8 por latin1

\usepackage[portuguese,english]{babel}
\usepackage{times}
\usepackage{url}
\usepackage{graphicx}
\usepackage{mdwlist}
\usepackage[nottoc]{tocbibind}
\usepackage{csquotes}
\usepackage[table,hyperref,x11names]{xcolor}
\usepackage{array,booktabs}
\usepackage{multirow}

\usepackage{dialogue}
%To get figures and tables side by side
\usepackage{floatrow}
\newfloatcommand{capbtabbox}{table}[][\FBwidth]
% end figures and tables side by side

%footnote in tables 
\usepackage{threeparttable}
%end footnote in tables

% Indice remissivo
\usepackage{makeidx}
\makeindex

%quotes
\usepackage{epigraph}
\usepackage{attrib}
\usepackage{titlesec}

%\titlespacing{\subsubsection}{0pt}{0pt}{0pt}
\titleformat{\subsubsection}[runin]{\normalfont\bfseries}{\thesubsection.}{3pt}{}


\usepackage[nomain,acronym,xindy,nonumberlist]{glossaries}
\newacronym{sdn}{SDN}{Software Defined Network}

\newacronym{nib}{NIB}{Network Information Base}

\newacronym{wan}{WAN}{Wide Area Network}

\newacronym{nos}{NOS}{Network Operating System}

\newacronym{os}{OS}{Operating System}

\newacronym{onf}{ONF}{Open Network Foundation}

\newacronym{of}{OF}{OpenFlow}

\newacronym{fifo}{FIFO}{First In,First Out}

\newacronym{smr}{SMR}{State Machine Replication}

\newacronym{arp}{ARP}{Address Resolution Protocol}

\newacronym{ip}{IP}{Internet Protocol}

\newacronym{mac}{MAC}{Media Access Control}


\makeglossary  

% Links
\usepackage{hyperref}

% Package para cabecalhos
\usepackage{fancyhdr}
\usepackage{lastpage}

\usepackage[font={small}]{caption}
\usepackage{subcaption}
\usepackage[sortcites=true, firstinits=true, isbn=false,
url=false, doi=false, eprint=false]{biblatex}

\bibliography{bibliografia,web}
\fancyhf{} %
\lhead{\nouppercase {\leftmark}} %
\rhead{\nouppercase {\bf \thepage}}
\renewcommand{\headrulewidth}{0.1pt}

% Comando para inserir pagina em branco (inserida na numeracao, mas sem
% numero impresso) para quando e' preciso obrigar um capitulo a comecar
% do lado direito (pagina impar)
\newcommand{\LIMPA}{
\newpage
\mbox{}
\thispagestyle{empty}
}

% Igual, mas insere pagina com numero impresso (normalmente nao se usa)
\newcommand{\LIMPAC}{
\newpage
\mbox{}
\thispagestyle{plain}
}


%%%%%%% PERSONAL COMMANDS %%%%%%%%%%%
\newcommand{\distcontrollers}{\cite{:vn, Tootoonchian:2010vy,Koponen:2010th,Yeganeh:2012jm}}
\newcommand{\distcontrollerspaper}{\cite{Tootoonchian:2010vy, Koponen:2010th,Yeganeh:2012jm}}
%END 

\newcommand{\tbl}[2]{\begin{tabular}{#1}#2\end{tabular}}
\newcommand{\ml}[2]{#1 $\pm$ #2} 

%
% ALTERAR AQUI AS INFORMACOES RELATIVAS AO PROJECTO
%
\newcommand{\PEITITULO}{A Consistent and Fault-Tolerant Network
  Information Base for Scalable Software Defined Networks}
\newcommand{\PEIAutor}{Fábio Andrade Botelho}
\newcommand{\PEIAutorNumAluno}{41625}

%Orientador e CoOrientador *sem* titulos (e.g. Prof. Doutor)
\newcommand{\PEIOrientador}{Alysson Neves Bessani}
\newcommand{\PEICoOrientador}{Fernando Manuel Valente Ramos} %se nao se aplicar, nao importa o que aqui esteja

%Se aplicavel, o supervisor pode ter um titulo (Dr., Eng.) colocado aqui
\newcommand{\PEISupervisorInstituicao}{Nome Completo do Supervisor}  %se nao se aplicar, nao importa o que aqui esteja

\newcommand{\PEIAnoLectivo}{2012/2013}
\newcommand{\PEIAno}{2013}

% Comentar/descomentar conforme conveniente
\newcommand{\PEITIPO}{DISSERTA\c{C}\~{A}O }
%\newcommand{\PEITIPO}{PROJECTO }

% Comentar/descomentar conforme conveniente
%\newcommand{\PEIIdiomaTese}{\selectlanguage{portuguese}}
\newcommand{\PEIIdiomaTese}{\selectlanguage{english}}

% Comentar/descomentar conforme conveniente
%\newcommand{\MEIEspecializacao}{Segurança Informática}
%\newcommand{\MEIEspecializacao}{Arquitectura, Sistemas e Redes de Computadores}
%\newcommand{\MEIEspecializacao}{Interac��o e Conhecimento}
%\newcommand{\MEIEspecializacao}{Engenharia de Software }

\usepackage{ifpdf}
\ifpdf
\pdfinfo {
	/Author (\PEIAutor)
	/Title (Projecto em Segurança Informatica)
	/Subject (Segurança Informatica)
	/Keywords (state machine replication, software defined networks)
	/CreationDate (D:20100510104905)
}
\fi

\usepackage[dvips]{geometry}
\geometry{a4paper=true,portrait=true,left=3cm,right=3cm,top=2.5cm,bottom=3.5cm}

\title{\PEITITULO}
\author{\PEIAutor}
%\date{\today}

\begin{document}

%Capa e pagina de rosto
\selectlanguage{portuguese}

\pagestyle{empty}

% ----------------------------------------------------------------------
% Capa
\begin{center}
\vspace{3cm}\normalfont\normalfont
\textsc{\huge{Universidade de Lisboa}}\\
\LARGE{Faculdade de Ci\^{e}ncias}\\
\Large{Departamento de Inform\'{a}tica}\\
\vspace{1cm}
\includegraphics[scale=.3]{pic/logo_ul.jpg}\\
%\includegraphics[scale=.6]{pic/logo_fcul.jpg}\\

\vspace{2cm}
\PEIIdiomaTese
\Large{\bf \PEITITULO}\\
\selectlanguage{portuguese}
\vspace{1cm}
%projecto realizado na\\
\vspace{0.7cm}
%\Large{\bf Nome da Instituicao de Acolhimento}\\
%\vspace{0.7cm}
%por\\
\vspace{0.7cm}
\Large{\bf \PEIAutor}\\
\vspace{2.4cm}
\Large{\bf \PEITIPO}\\
\vfill
\Large{MESTRADO EM ENGENHARIA INFORM\'{A}TICA}\\
\large{Especializa\c{c}\~{a}o em \MEIEspecializacao}\\
%\vspace{1.5cm}
\vfill
\PEIAno
\end{center}
\newpage
\mbox{}
\newpage
% Fim da capa
% ----------------------------------------------------------------------

\setcounter{page}{1}
\pagenumbering{roman}

% ----------------------------------------------------------------------
% Folha de Rosto

\begin{center}
\vspace{3cm}\normalfont\normalfont
\textsc{\huge{Universidade de Lisboa}}\\
\LARGE{Faculdade de Ci\^{e}ncias}\\
\Large{Departamento de Inform\'{a}tica}\\
\vspace{1cm}
\includegraphics[scale=.3]{pic/logo_ul.jpg}\\
%\includegraphics[scale=.6]{pic/logo_fcul.jpg}\\
\vspace{2cm}
\PEIIdiomaTese
\Large{\bf \PEITITULO}\\
\selectlanguage{portuguese}
\vspace{1.5 cm}
\Large{\bf \PEIAutor}\\
%\vspace{2.4cm}
\vspace{2 cm}
\Large{\bf \PEITIPO}\\
\end{center}
\vspace{0.5 cm}
\begin{center}
\Large{MESTRADO EM ENGENHARIA INFORM\'{A}TICA}\\
\large{Especializa\c{c}\~{a}o em \MEIEspecializacao}\\
\end{center}
\vspace{1 cm}
Disserta\c{c}\~{a}o orientada pelo Prof. Doutor \PEIOrientador \\
% DESCOMENTAR a linha relevante (se alguma), removendo o % no inicio
%e co-orientado pelo Prof. Doutor \PEICoOrientador \\
%e por \PEISupervisorInstituicao
\vfill
\begin{center}
\large\PEIAno
\end{center}
\newpage
\thispagestyle{empty}
\mbox{}
\newpage
% Fim da Folha de rosto
% ----------------------------------------------------------------------


% Agradecimentos
\pagestyle{plain}

\vspace*{2cm}
\begin{center}
\selectlanguage{portuguese}
\Large \bf Agradecimentos
%\selectlanguage{english}
%\Large \bf Acknowledgments
\end{center}
\vspace*{1cm} \setlength{\baselineskip}{0.6cm}

 Á minha familia, que mais do que ser familia, acreditou em mim, e levou-me às costas demasiados anos. Eu nunca vos vou conseguir retribuir o quanto fizeram por mim depois de tanto abismo cruzado, e de tanta rotunda demorada. As saídas nem sempre são óbvias à partida, independente da via de entrada escolhida. Pacientemente cheguei lá por  mim, mas nunca seria possível sem todos os vossos sacríficios e paciência. De seguida à minha  companheira de viagem, Maria Lalanda, por me acompanhar nesta jornada comprida. Foste o melhor deste caminho, espero no futuro perder-me apenas contigo. E nunca te esqueças que não chove em Santiago. 

Aos meus orientadores: Alysson,  Ramos e Diego Kreutz. Um muito obrigado por toda a paciência, apoio, e confiança ao longo do meu último ano. Sinto que estive muito acquém do que devido, por mais esforço que tenha colocado neste projecto. Espero ter aprendido com os erros, e espero que no restante processo desse trabalho, possa-vos supreender com mais e melhor. Também quero agradecer por ao prof. Alysson por me ter ensinado o \\vspace (prometo não sobre-utilizar :) e ao prof Fernando Ramos por me ter vendido o meu proprio peixe. 


Queria agradecer também aos reviewers da EWSDN pelos comentários que ajudaram a melhor o paper que foi resultado deste trabalho. Além disso queria agradacer ao financiamento que permitiu o meu trabalho ( parcialmente supportado pelo EC FP7 através do projecto BiobankCloud (ICT- 317871) e também pela FCT através do programa Multi-anual do LASIGE). 

De restante queria agradecer a uma grande lista de pessoas e coisas. Em especial: 
Ao pessoal do Jardim, pelas baldas, pelo jardim, por todo o tempo envoltos em ritmo e poesia. São demasiados os nomes para referir aqui, mas não podia deixar de referir as pessoas que por cirscunstância tiveram mais impacto em eu estar onde estou: João Sardinha, e Tâmara Andrade. Os restantes não deixam de ser igualmente importantes para mim. Á Catarina e Filipe Rebelo, e à senhora Eleanor por terem acreditado em mim.  Ao Manuel Barbosa por me alavancar em Braga, e ao  José Silva e à Barbara por se infiltrarem na minha casa :) .   Á nossa casa, e a voçês três por termos constituido familia. Nunca vou esquecer aquele espaço, apenas por causa da voçês. Á Senhora Conceição também. 


Aos Monads na janela, ao Pingo Doce, Lidl, Pizarria dos lux e a todas as máquinas automáticas do c5 e a do c6. Houvesse trocos em todas aquelas noitadas.... Ao café, ao SG Ventil, aos seguranças que estavam sempre em ronda e não me deixavam ir fumar : um não agradecimento. Ao gadgets do Tocha, ao quadro da residencia, à vista sobre a 25 de abrirl, e a todos os hipsters da príncipe real. Ao miradouro, ao jardim e ao quiosque. À Pizarria do Lux, a sério. Á máquina de café da residencia. Ao stack-overflow e ao lego-coding. 


% Ramos who taught how to sell fish. 
% Allyson who taught me about \vspace. I promise not to overuse it .  
%sá, we will eventually work again. 
%last but not least, to Leslie  Lamport, for wearing t-shirts on presentations,  inventing LaTex and a zillion of fundamental papers. 

%TO haskell who taught me to think. To java and c for making me unlearn that. God forbid (I will ever be a programmer!) 
Á longuria, ao insularismo e  á Ilha! 
E por último, mas não menos importante (olha o cliché!), ao Charlie que conseguiu sempre sair da selva. Mesmo sem cão. 

And last, but not least: to Charlie, who has always made out of the jungle! Event without a dog! 
\LIMPA
\LIMPA

~
\vfill

\selectlanguage{portuguese}
%\selectlanguage{english}
\begin{flushright}\textit{A todos ``os meus conhecidos''  que já levaram porrada. }\end{flushright}

\LIMPA

%%% Local Variables: 
%%% mode: latex
%%% TeX-master: "../PEI"
%%% End: 


% Pagina do resumo em portugues
%\pagestyle{empty}

% ----------------------------------------------------------------------
% P�gina do resumo em Portugu�s:
\selectlanguage{portuguese}
\vspace*{2cm}
\begin{center} \Large \bf Resumo
\end{center}
\vspace*{1cm} \setlength{\baselineskip}{0.6cm}

Os documentos escritos em Portugu\^{e}s devem ter um resumo em Portugu\^{e}s e um resumo noutra l\'{i}ngua comunit\'{a}ria que contenham at\'{e} 300 palavras cada. Quando o conselho cient\'{i}fico autorizar a apresenta\c{c}\~{a}o do trabalho final escrito em l\'{i}ngua estrangeira, este deve ser acompanhado de um resumo adicional em Portugu\^{e}s de, pelo menos, 1200 palavras.

Lorem ipsum dolor sit amet, consectetuer adipiscing elit, sed diam nonummy nibh euismod tincidunt ut laoreet dolore magna aliquam erat volutpat. Ut wisi enim ad minim veniam, quis nostrud exerci tation ullamcorper suscipit lobortis nisl ut aliquip ex ea commodo consequat. Duis autem vel eum iriure dolor in hendrerit in vulputate velit esse molestie consequat, vel illum dolore eu feugiat nulla facilisis at vero eros et accumsan et iusto odio dignissim qui blandit praesent luptatum zzril delenit augue duis dolore te feugait nulla facilisi. Nam liber tempor cum soluta nobis eleifend option congue nihil imperdiet doming id quod mazim placerat facer possim assum. Typi non habent claritatem insitam; est usus legentis in iis qui facit eorum claritatem. Investigationes demonstraverunt lectores legere me lius quod ii legunt saepius. Claritas est etiam processus dynamicus, qui sequitur mutationem consuetudium lectorum. Mirum est notare quam littera gothica, quam nunc putamus parum claram, anteposuerit litterarum formas humanitatis per seacula quarta decima et quinta decima. Eodem modo typi, qui nunc nobis videntur parum clari, fiant sollemnes in futurum. Lorem ipsum dolor sit amet, consectetuer adipiscing elit, sed diam nonummy nibh euismod tincidunt ut laoreet dolore magna aliquam erat volutpat. Ut wisi enim ad minim veniam, quis nostrud exerci tation ullamcorper suscipit lobortis nisl ut aliquip ex ea commodo consequat. Duis autem vel eum iriure dolor in hendrerit in vulputate velit esse molestie consequat, vel illum dolore eu feugiat nulla facilisis at vero eros et accumsan et iusto odio dignissim qui blandit praesent luptatum zzril delenit augue duis dolore te feugait nulla facilisi. Nam liber tempor cum soluta nobis eleifend option congue nihil imperdiet doming id quod mazim placerat facer possim assum. Typi non habent claritatem insitam; est usus legentis in iis qui facit eorum claritatem. Investigationes demonstraverunt lectores legere me lius quod ii legunt saepius.

%Para documentos em Portugu�s: Resumo em portugu�s at� \textbf{300} palavras.
%Para documentos em l�ngua estrangeira: Resumo em portugu�s com pelo menos \textbf{1200} palavras.

\vfill

\begin{flushleft}
\textbf{Palavras-chave:}
cerca de 5 palavras-chave
\end{flushleft}

\LIMPA
% Fim da p�gina do resumo em Portugu�s
% ----------------------------------------------------------------------


% Pagina do resumo em ingles
% ----------------------------------------------------------------------
% P�gina do resumo em Ingl�s:
\selectlanguage{english}
\vspace*{2cm}
\begin{center}
\Large \bf Abstract
\end{center}
\vspace*{1cm} \setlength{\baselineskip}{0.6cm}

This work is motivated by the 
emergent network architecture of Software
Defined Networks where the control of the network is extracted from the
network devices and delegated to a
server named controller that is responsible for dynamically
configuring the network devices present in the infrastructure. The
controller has the advantage of logically centralizing the network
state in contrast to the previous model where state was distributed
across the network devices. If spite of this logical centralization,
the control plane (where the controller operates)  must be distributed
in order to support the scale of modern datacenters and
networks. However, this distribution introduces several challenges due
to the heterogeneous, asynchronous and faulty environment where the
controller operates. Current distributed controllers
lack transparency due to the  eventual consistency
properties employed in the distribution of the controller. This results in a
complex programming model for the development of network control applications. This work will
contribute with a fault-tolerant distributed controller with strong
consistency properties that allows a more transparent distribution of
the control plane. Also we believe that even if favoring 
consistency, we will be able to provide superior performance results
that those available in the literature. 


% The success of the current Internet infrastructure and architecture is
% indisputable. However, it  has long suffered from negative
% criticism. The "classic'' Internet architecture suffers from serious
% drawbacks in the management field as the inherent complexity
% associated with the manual configuration of  distributed 
% network equipment can be too much to bare with only rudimentary
% mechanisms as command line interfaces and hand-made configuration
% scripts. Human error is known to be the most probable cause of network
% downtime. Additionally the lack of control abstractions  in
% networking  prevents the deployment of both new and modified control functions.

% These problems are cumulative as both network designers
% and administrators struggle to satisfy modern network
% requirements. Currently several control mechanisms are designed,
% deployed and configured in isolation of each other. Known examples
% are:intra domain routing; inter domain routing; access control; QoS services; packet inspection;load-balancing; and intrusion detection. These control mechanisms
% are deployed in network devices such as routers and switches with no
% support for their modification. 
% Thus, devices that should  be optimized for data forwarding end up bundled with
% heavyweight control features that must be configured independently. 

% Software Defined Networking (SDN) radically changes the current IP
% architecture by extracting the control function from the network
% devices and shifting it to a centralized service known as the control
% plane. This plane, with the help of a coherent view of the network state
% can dynamically configure the network devices.
% The network state
% is centralized in a datastore known as Network Information Base (NIB)
% or View. The NIB  can provide up
% to date connectivity, namespace information, and other physical or
% logical information capable of satisfying network control  objectives. 
%  The result configuration is the output
% of a function applied to the current network state. In SDNs networks devices are
% only responsible  for packet forwarding  as opposed  to the classic
% Internet architecture where devices also run distributed algorithms
% in order to perform self-configuration. 

% Centralized control planes have been reported to scale to tens of
% thousands of hosts. However the current
% state of the art implementations are still far from satisfying the
% requirements of modern datacenters and Wide Area Networks as the high traffic present in these networks may lead a
% centralized service to exhaustion. In addition latency issues presented
% in  WAN's covered by a centralized control planes may not
% be acceptable under the quality of service desired. Finally a single
% point of failure is not an usual option  as the failure of the control plane
% may compromise network availability. All this  strongly
% suggests the distribution of the control plane. However, the
% distribution processes must not affect the benefits of the
% SDN architecture even if operating in a heterogeneous, asynchronous and faulty
% environment.

% Although SDN has been considered a hot
% research topic the existing work does not covers distribution of the
% control plane extensively. Two designs are often-cited: HyperFlow \cite{Tootoonchian:2010vy} and
% Onix \cite{Koponen:2010th}. The latter replicates network events that cause changes to the NIB
% while the former replicates and distributes the NIB state. We consider both those works to have serious
% drawbacks as lack of distribution transparency and weak consistency
% properties leading  to a complex programming model for
% the development of network control applications. 

% In our work we plan  to contribute to the Software
% Defined Network field by introducing a strong consistent Network
% Information Base integrated in an open source SDN controller. Our NIB
% is replicated across controllers instances in a distributed state machine
% fashion that we implement with  well-known protocols to the  Distributed
% Systems field. We believe that the Network community has overlooked
% both the correctness and simplicity advantages of strong consistent datastores as
% well as their performance. We
% also belief that correctness problems can arise from the use of 
% weak consistency datastores if the application is not fully aware of
% the distribution guarantees given by the NIB. 


%Resumo at� \textbf{300} palavras. 

\vfill

\begin{flushleft}
\textbf{Keywords:}
Replication, Strong Consistency, Distributed State Machine,
Distributed Control Plane, Software Defined Networking. 
\end{flushleft}

\LIMPA
\selectlanguage{portuguese}
% Fim da p�gina do resumo em Ingl�s.
% ----------------------------------------------------------------------


\pagestyle{plain}

\PEIIdiomaTese

% Indice
\tableofcontents

\LIMPA

%Lista de figuras
\listoffigures

%\addcontentsline {toc} {chapter} {Lista de Figuras}
\newpage
\thispagestyle{empty}
\mbox{}
\newpage

%Lista de tabelas
\listoftables

%\addcontentsline {toc} {chapter} {Lista de Tabelas}
\newpage
\thispagestyle{empty} \mbox{}
\newpage

% ----------------------------------------------------------------------
% Inicio conteudo
\pagestyle{fancy}
\cleardoublepage

\setcounter{page}{1}
\pagenumbering{arabic}

\chapter{Introduction}
\glsresetall
% contexto do trabalho
% se resume o trabalho desenvolvido
% se identificam  contribuicoes 
% estrutura do relatorio 
% sucintamente o enquadramento instituicional 

%Outro capitulo? 
% pormenor os objectivos do trabalho
% contexto subjacente 
% metodologia  utilizado no desenvolvimento como o planeamento efectuado para o concretizar 
% Apresentar uma confrontacao com o plano de trabalho inicial  analisando as razoes de eventuais desvios occorridos


%``With the strong consistency model users are not able to discover that data is replicated. In fact the client of systems that provide this model are not able to distinguish him in any way from a centralized system.'' This is a natural way of thinking. 

%``similar to the examples used to motivate SDN today. The issues of the day included network service provider frustration with the timescales necessary to develop and deploy new net- work services (so-called network ossification), third-party interest in value-added, fine-grained control to dynam- ically meet the needs of particular applications or net- work conditions, and researcher desire for a platform that would support experimentation at scale. Additionally, many early papers on active networking cited the prolif- eration of middleboxes, including firewalls, proxies, and transcoders, each of which had to be deployed separately and entailed a distinct (often vendor-specific) program- ming model. Active networking offered a vision of uni-''

%TODO - tens que dizer logically centralized pode ser fisicamente distribuído. 

%TODO - decomposed network goals - given an example.  Routing protocols handle dynamic topology changes but have to be configured individually (per device). Furthermore they do not work in concert witht other control functions such as firewalls implement as packet filters statements applied to each interface. As an example suppose that a router $r_i$ has two interfaces $i_1$ and $i_2$ and a packet filter statement denying traffic from a particular network $n$ . Then if the vent that   It is common to found that the latter can not be dynamically adapted to topology changes. 

%Distributed state management. Logically centralized route controllers faced challenges involving distributed state management. A logically centralized controller must be replicated to cope with controller failure; however, inconsistencies can arise between the controller replicas. Researchers explored the likely failure scenarios and consistency requirements. At least in the case of routing control, the controller replicas did not need a general state management protocol, since each replica would eventually compute the same routes (after learning the same topology and routing information) and transient disruptions during routing-protocol convergence were acceptable even with legacy protocols [12]. This problem would arise again five years later in the context of distributed SDN controllers. \emph{Yet, distributed SDN controllers face the far more general problem of supporting arbitrary controller applications, requiring more sophisticated solutions for distributed state management.}


%DO NOT FORGET TO TALK ABOUT DEALING WITH ROUTING PROBLEMS IS EVENTUALLY CONSISTENT. 

%Use dijktra not bellman-ford 
%``This centralized paradigm is more flexible, since new functionality can be centrally programmed at the decision element that requireding a new distributed algorithm, but raises the specter of a single point of failure. However adequate resilience can be achieved by appling standard replication techniques to the decision element. The replication techniques are completely decoupled from the network control algorithms, so they do not impede application innovation''. 


\section{Software Defined Network}
Despite its success, current \gls{ip} networks suffer from problems which have long drawn attention of the network academic community. 
Researchers tackle those problems by following one of two strategies: the first is focused in tailoring the performance of \gls{ip}  based networks and/or providing ad hoc solutions to new technological requirements, while the second  advocates the clean slate redesign of the \gls{ip} architecture. 
\gls{sdn} is the pragmatic result of several contributions to the clean slate strategy that has the benefit of radically transforming the management and specification of \gls{ip} networks albeit maintaining intact the existent \emph{host-to-host} protocols (i.e., the TCP/IP stack\footnote{The TCP/IP stack is a common name for the Internet protocol suite comprising a networking model and set of communication protocols used in the Internet.}). 

%Two approaches have been followed in addressing those problems: the first is focused in tailoring the performance of \gls{ip}  based networks and/or providing ad hoc solutions to new technological requirements, while the second  advocates the clean slate redesign of the \gls{ip} architecture. 

%\gls{sdn} is the pragmatic result of several contributions to the clean slate approach and has the advantage of radically change the way \gls{ip} networks are managed and developed without requiring changes to the \emph{host-to-host} protocols~\cite{ONF:2012ui}. 
In a nutshell, \gls{sdn} shifts the control logic of the network (e.g., route discovery) from the network equipment to a commodity server where network behavior can be defined in software from a single high-level control program, without the constraints set by the network equipment. Thus, there is a separation of the control plane, where the server operates, from the  data plane where the network infrastructure resides. 
A fundamental abstraction in \gls{sdn} is the concept of \emph{logical centralization} that specifies that the control plane operates with a logically centralized global view of the network, which can contain (among other things)  the topology, link state, and security policy. 
This global view enables simplified programming models and facilitates network applications design.

A logically centralized programmatic model does not postulate a centralized system. Arguably, a less-prone-to-ambiguity definition for ``logically centralized'' could be ``transparently distributed'' because \emph{``either you're centralized, or you're distributed''}\,\cite{casado2011}.
In fact, the need to guarantee adequate levels of performance, scalability, and reliability preclude a fully centralized solution.
Instead, production-level \gls{sdn} network designs resort to physically distributed control planes.
Consequently, the designers of such systems have to face the fundamental trade-offs between the different consistency models, the need to guarantee acceptable application performance, and the necessity to have a highly available system.

% Distribution of the control plane is argued by many, to be essential for scalability and reliability reasons \cite{Tootoonchian:2010vy, Koponen:2010th,Yeganeh:2012jm,:zr}. However, state of the art  distributed control planes lack transparency and consistency proprieties that are desirable in the development of network control applications. 
% Also we believe that, based on the state of the art replication results available \cite{Rao:2011vz,Lee:1996jm,Bolosky:2011ve,Wang:2012tj} we can contribute to SDN with an efficient distributed control plane.
% Finally, we also believe that Software Defined Networking will have significant deployment support from the network industry and may eventually take a primary role in the replacement of the current IP architecture. 
% All these factors  encourage us to pursue the work described in this document.

In this document, we propose a novel SDN controller architecture that is distributed, fault-tolerant, and strongly consistent.
The central element of this architecture is a data store that keeps relevant network and applications state, guaranteeing that SDN applications operate on a consistent network view, ensuring coordinated, correct behavior, and consequently simplifying application design. The drawback of  this approach  is the decrease of performance, which limits responsiveness and hinders scalability. Even assuming these negative consequences, an important conclusion of this study is that it is possible to achieve those goals while maintaining the performance penalty at an acceptable level.

\subsection{Standard Network Problems}
%network devices are configured indepentenly in a low-level device specific manner can be very buggy and unpredictable. To cope with this 
%Much easier to coordinate, network operator can write a program that allows the behaviour of network  

%It is hard to evolve because the software is bundle with the hardware. It is much easier to evolve when the software is decoupled from the hardware. 


Classic \gls{ip} networks are complex to manage and control. 
%This complexity is drived  by different factors.  Namel
Researchers argue that this complexity descends from the integration of network control functionalities such as routing 
in the network devices that should be solely responsible for forwarding packets. 
For example, both Ethernet\footnote{A family of computer networking technologies for local area networks.}  switches and \gls{ip} routers are in charge of  data forwarding as well as path computation\footnote{Paths are found by processing routing protocols in the \gls{ip} case and by processing the \gls{stp} protocol in the Ethernet case.}. 
In fact, control logic is responsible for  tasks that go beyond path computation such as  discovery, tunneling, and  access control. 
This vertical integration of control and data plane functions is  undesirable for multiple reasons:

First, it requires multiple ad-hoc distributed protocols to implement a myriad of control functions which are arguably more complex than centralized control based on a global view of the existent network state.  
This problem is aggravated if we consider the scale of existent networks. 
To make matters worse, each of those control functions commonly works in isolation. 
To exemplify, the topology process does not cooperate with the access control process, thus making it possible that a link failure causes a breach in the security policy that must be solved by human intervention. 

Second, control functions occupy resources from the network equipment requiring more powerful and expensive hardware. 
The path computation process itself is so complex that it may require devices to concurrently execute several control processes and/or maintain a global view of the network (in each device!).  

Third, it difficulties network innovation since new network protocols must undergo years of standardization and interoperability testing. 
This protocols are then implemented in closed, proprietary software and cannot be leveraged, modified, or improved upon. 
As such researchers have difficulty in testing and deploying new control functionality in real networks. 


Finally, the configuration of the data plane requires low level and device-dependent instructions through direct command line interaction or ad-hoc scripts. 
This process is error-prone. 
%This process of translating high-level network objectives by humans to low level device-dependent configuration primitives is error prone. Furthermore, configuration errors (that should be insignificant events) can cascade into network errors of a global scale. 
As an example, in 2008 Pakistan Telecom took Youtube offline almost worldwide by following a censorship order~\cite{McCullagh:2008fk}.
Another consequence of this complexity is that administrators of large networks  must have significant human resources for the management task.  

Ultimately, mingling the data plane and control plane into a vertical integrated, proprietary solution has cause networks to be difficult to manage, hard to innovate, and expensive to maintain. 
%TODO - problematic since? pe

% Finally, control functions are isolated of one another and lack
% abstractions for consulting, modifying and configuring their state at
% run-time. As a practical example consider the case of access control
% in a network. For this purpose we can associate packet filtering
% statements to each router interface whereby we
% explicitly specify the traffic allowed or forbidden. In this scenario
% if the network topology changes, the packet filtering rules installed
% may violate  the network security policy. As the rules configuration
% can not be made directly dependent of the topology due to the
% isolation of control abstractions, we must manually change them in
% order to satisfy the safety requirements of the security policy.  


%and innovate upon.  
\subsection{Logical Centralization} 
\gls{sdn} is a novel architecture that emerged from the drawbacks set by closely-coupling the control and data planes. 
Fig.~\ref{fig:sdn.2d}(a) shows that this architecture  physically decouples those planes. 
In \gls{sdn} all network control functionality  such as routing, access control, load balancing, etc. can be defined in software and performed by a controller with the help of a logically centralized \emph{Network View} containing  all the relevant network state (e.g., network topology, forwarding tables, acess control). 
This state can be present in the controller memory or in  a \emph{datastore}  to which the controller has access to. 
%Throughout the text we refer to this datastore as the Network Information Base (NIB) or \emph{view} as can be seen in the figure. 

%This process is shown in Figure \ref{fig:sdn.2d}(b), with the network state represented as the NIB. Function $f$ can be dynamically invoked when changes occur in the NIB, which may result in a different configuration of the data plane. In the light of this model we can see that the NIB takes a fundamental role in the actions taken by the controller. 

%TODO - podes dizer que os controladores centralizados de pipeline são um género de f(NIB, PACKET_IN). 

\begin{figure}
  \centering 
  \footnotesize
  \includegraphics[width=\textwidth]{pic/intro/decoupled}
  \caption[SDN architecture]{SDN architecture: the controller maintains a connection to the network devices residing in the data plane. The \emph{Network View} contains all the relevant network state (e.g., topology information) and configuration (e.g., access control). The configuration of network devices is determined based on this state.}
  \label{fig:sdn.2d}
\end{figure}

In order to separate the control plane from the data plane it is crucial that the latter implements an interface that allows  the configuration of the network devices. 
\gls{of} is the most common protocol that implements this interface~\cite{openflow}.
In \gls{of} forwarding is based on flows that are broadly equivalent to a single host-to-host network interaction. 
%--- a sequence of packets with specified headers in common. 
The control control plane  manipulates the flow-based forwarding tables present in network devices. 
Devices recognize flows (e.g., \emph{any tcp packet destined to port 80} or \emph{any ip packet destined to 1.1.1.1}) and associate them with actions (e.g., \emph{drop packet}, \emph{forward to port $x$}, \emph{forward to controller}). 
Matching flows have associated actions as opposed to unmatching flows that can be  forwarded to the controller. 
This method has the advantage of possibly redirecting only one packet per flow to the controller (which benefits scalability). 
In the  \gls{sdn} architecture it is also vital that the \emph{Network View} is constantly updated. 
In \gls{of} this is accomplished by events triggered from the data plane to the control plane containing new updates, relevant to the network state,  such as topology changes (e.g., \emph{link up}, \emph{link down}).  

In general, the objective of the control plane can be seen as to implement a function $f$, representing all control functionality,  that has the \emph{Network View}  as  input and the configuration of network devices as output. 
Alternatively, (and most common) the devices can request forwarding advice to the controller for packets on the fly. 
 In this case, function $f$  is expanded to receive both the network state and the specific packet as input, and the output determines  a viable configuration for forwarding that given packet and possibly others alike. 


The fact that configuration is now software driven, coupled with the logical centralization of the network state, allows simple and automatic deployment, configuration and development of  control functionalities. 
Furthermore, as opossed to the previous condition of networks, \gls{sdn} enables standard techniques and approaches to fundamental networking problems that were once difficult to apply due to the closed nature of the equipment. 

It is worth pointing out that Software Defined Network is not just an  artifact for the scientific community, but it is also being adopted  by the industry. For example, Google has deployed a Software Defined  WAN (Wide Area Network) to connect their datacenters  \cite{Hoezle:2012uq}. 
Additionally, this company and other industry  partners (Yahoo, Microsoft, Facebook, Verizon and Deutsche Telekom,  Nicira, Juniper, etc.) have formed the Open Network Foundation (ONF)~\cite{onf} --- a non-profit organization  responsible for the  standardization process of SDN technology. 
Finally, several network  hardware vendors currently support OpenFlow.
Examples  include IBM, Juniper, and HP. 

\subsection{Distributed Control Plane}
Most SDN controllers are centralized, leaving to its users the need to address several challenges such as scalability, availability, and fault tolerance.
However, the need for distribution has been motivated recently in the SDN literature. 
In fact, distributed controllers already exist \distcontrollers.  

We present the following arguments supporting the distribution of the control plane: 

\begin{itemize}
\item[] \textbf{Scalability:}  The controller resources (memory and CPU) do not support all network sizes and dynamics.  Memory contains the network state and the CPU is used for processing network events - mainly new flow events. The use of both these resources grows linearly with the size of the network managed by the control plane eventually leading to  resource exhaustion. Thus, a scalable control plane requires the distribution of the network state and/or flow processing; 
\item[] \textbf{Performance:}  Scalability may partially be considered for performance reasons also. However, there are more intransigent performance requirements such as latency in \glsplural{wan} where big latency penalties may be observed between the  control and data plane  communication; 
\item[] \textbf{Fault Tolerance:} Network control applications built in the controller  may require the availability and durability of the service. Even if failures in the control plane are inevitable, it is desirable to tolerate those without disrupting the network. 
\end{itemize}

%TODO - set the number on switches. 
\textbf{Scalability} is a fundamental reason for distributing. 
Although centralized controllers have been reported to handle: tens of thousands of hosts~\cite{Casado:2007kb}; a million events per second averaging 2.5ms per event~\cite{controllerPerformance}; and millions of flow requests per second~\cite{beacon} there are limits in resources that will eventually lead to their exhaustion. 
These limits are easily reached in current data centers and \glsplural{wan}. 
Namely, there is evidence of data centeres that can  easily reach thousands of switches and hundred of thousands hosts~\cite{Scott:2012tt}. 
Also Benson~\etal\ show that a data center with 100 edge\footnote{The three tier data center topology is an hierarchical topology with 4 levels. In the bottom levels reside the application servers connected to the edge switches.} switches can, in the worst case, have spikes of 10M  flow arrivals per second~\cite{}. 
These numbers strongly  suggest distributing the control plane in order to shield controllers from a large number of network events. 

The \textbf{performance} reason presented may not be the only one, but it is fundamental. 
At the time of writing only one \gls{sdn} enabled \gls{wan} is known \cite{jain2013b4}, but given its publicized success one could expect more to follow. 
Even though the control plane only requires processing the first packet in flow arrivals, the latency established in this communication must be minimal such that network applications are not noticeable affected. 
Distribution can mitigate the latency problem by bringing the control plane closer to the data plane. 
%TODO - controller placement problem finds that only one controller is enough,  

Finally,  \textbf{Fault tolerance} is an essential part of any Internet-based system, and this property is therefore typically built-in by design. 
Solutions such as Apache' Zookeeper (Yahoo!), Dynamo (Amazon)  and Spanner (Google)  were designed and deployed in production environments to provide fault tolerance for a variety of critical services.
The increasing number of SDN-based deployments in production networks is also triggering the need to consider fault tolerance when building SDNs.
For example, Google presented recently the deployment of its inter-datacenter WAN with centralized traffic engineering using SDN and OpenFlow.
Such centralized control requires (and employs) fault tolerance.

In summary, there is a pleothora of reasons for distributing the control plane. However there several challenges in doing it so. 
However until today, due the state dependencies present between distributed controllers there is no evidence of scalable distributed control planes (when considering the number of events processed). 
Arguably, it might be the case that this is due to the state dependencies between different controllers in a distributed control plane. 
Furthermore, the generality of a control plane antecipates the usage of arbitrary controller aplications that will require different and general distribution mechanisms eagaring for scalability and performance while maintaing reliability.
Truly this is a hard challenge to tackle. 

\subsection{Distribution Tradeoffs}
Fundamental principles such as the  Brewer's CAP theorem \cite{Brewer:fk} (formally proven by Gilbert and Lynch \cite{Gilbert:2002il}) are an example of classic distribution tradeoffs. This theorem states that a distributed system may be simultaneously (i.e., at the same time) qualified with only two of the following properties: consistency, availability and partition-tolerance. Furthermore, since supporting partition-tolerance implies tolerating arbitrary message loss  \cite{Gilbert:2002il},  the vast majority of systems are forced to choose between availability and consistency. This choice also applies to the distribution of the control plane. 


The tradeoffs associated with this choice have already been studied by Levin et al~\etal\ in the context of SDN~\cite{levin}. This work found that two control applications (load balancers) that differ in the choice between availability and consistency have different optimality results. Levin et al. shows also that when availability is chosen the control application optimality5 is significantly degraded. In spite of not all control applications effectively have optimality requirements that are affect by the choice, similar tradeoffs arise in other applications such as: firewalls, intrusion detection and routing.

As well as degrading optimality, favoring availability also affects transparency - the property of a distributed system to appear as a centralized service to its users. To explain, the choice of availability implies that the user of the system must be aware of the lack of consistency. Thus, if correctness or optimality are relevant for the user, additional effort and reasoning is required in the interaction with the service. 

Currently, often-cited distributed control planes as Onix \cite{Koponen:2010th} or HyperFlow  \cite{Tootoonchian:2010vy} favor availability over consistency. To be true Onix allows the choice to be made between availability and consistency for different control applications. However this choice reveals a less transparent and complex API that forces users to reason about correctness and problems that may arise from the inconsistency of data in the distribution process. 

In our work we favor consistency and consequently transparency in the distribution of the control plane being that we believe that it can simplify application development and it can expose better results that ones portraited in \gls{sdn} literature.
One of our goals is in fact to show that, despite the costs of consistency, the \emph{``severe performance limitations''} (quote from\,\cite{koponen2010}) reported for Onix's consistent  data store are a consequence of the particular implementation and not an inherent property of these systems. 


%In summary, it is usual in distributed systems to choice between consistency and availability. While the latter is informally associated with correctness and latency penalties, the former is associated with incorrectness and scalability. 
 %TODO - can do better. Tradeoffs between scalability, performance and fault tolerance. 
%  Even if \textbf{scalability} is necessary it is not straightforward what is the best way to adopt it. The distribution method choosen must lower the usage of resources and not raise it. For this reason a controller instance must shield event processing from other controllers as well as local state. Equally problematic is the \textbf{performance} issues associated with latency. If we distribute for minimal latency in control to data plane communication we have to minimize the inter-controller communication as it can easily present latency penalties. Finally \textbf{fault tolerance}  presents challenges as it is necessary to perform failure detection and take appropriate measures to take over one controller functions without disrupting the network functional behavior and correctness. The usual solution is to replicate, but this  may affect the scalability of the controller. 

% Other challenges are associated with the distribution of the control plane such as correctness and simplicity of network management applications built on top of it. Consider, as an example, a simple SDN application that contributes to the Network Information Base with high-level namespaces bindings (e.g., user names and associated dynamic IP). Intransigent applications requirements such as logging and access control enforcement then it is strictly necessary to guarantee  that the failure of controllers does not implies the loss of any information currently being tracked. As another example a distributed control plane is more prone to correctness errors as loops or even black holes caused by inconsistency of the network state. Inconsistency, even if  transient, may cause cyclic or conflicting network paths. Notice however, that correctness of applications may always be guaranteed if applications are fully aware of the distribution mechanisms employed by the control plane.\\
%TODO - inconsistent vs consistentcy tradefoof. 
% \subsection{State Machine Replication}
% TO BE WRITTEN: MOTIVATED BY CURRENT PERFORMANCE OF STATE MACHINE REPLICATION. 


Recent work on SDN has explored the need for consistency at different levels. 
Network programming languages such as Frenetic\,\cite{Foster2011} offer consistency when composing network policies (automatically solving inconsistencies across network applications' decisions). 
Other related line of work proposes abstractions to guarantee data-plane consistency during network configuration updates~\cite{reitblatt2012abstractions}. 
The aim of both these systems is to guarantee consistency \textit{after} the policy decision is made. 
Onix\,\cite{koponen2010} provides a different type of consistency: one that is important \emph{before} the policy decisions are made. 
Onix provides network state consistency --- both weak and strong --- between different controller instances. 
The data store we propose is similar in that it offers strong consistency for network (and application) state between controllers\footnote{By strong consistency we mean that any read that follows a write will see the result of such write.}. 
One of our goals is in fact to show that, despite the costs of consistent replication, the \emph{``severe performance limitations''} (quote from\,\cite{koponen2010}) reported for Onix's transactional persistent database are a consequence of the particular implementation of such data store, and not an inherent property of these systems. 


\section{Goals and Contributions}
%REFUTABLE CLAIMS WITH FORWARD REFERENCES 

Our work has the single objective of enabling distribution of a control plane through the state distribution of the network state. 
For this we built a simple key-value data store on top of a state-of-the-art replication middleware that enables both fault-tolerance and consistency. 
Afterwards we integrated this data store with a centralized controller and modified different applications to use the this controller. 

In this thesis we argue that it is possible, using state-of-the-art consistent replication techniques, to build a distributed SDN controller that not only guarantees strong consistency and fault tolerance, but also does so with acceptable performance for many SDN applications.
In this sense, the main contribution of this thesis is showing that if a data store built using such techniques (e.g., as provided by BFT-SMaRt~\cite{smart-tr, bft-smart:2011:High-perfomance}, a high-performance fault-tolerant state machine replication middleware) is integrated with a production-level controller (e.g., Floodlight~\cite{flood}), the resulting distributed control infrastructure could handle efficiently many real world workloads.

%The objective of the experiments covered in this chapter  is to analyze the workloads generated by these applications to thereafter measure the performance of the data store when subject to such realistic demand caused by real applications.

The contributions can be summarized as following: 
\begin{itemize}
\item A distributed controller design, exhibiting fault-tolerance and transparency that can be built by expanding on existent centralized controller platforms(chapter~\ref{});
\item A study of the workloads produced by three production-level controller applications on a data store built for the distribution of the network state (chapter~\ref{}); 
\item The evaluation of  a state-of-the-art replication middleware capability to process the workloads mentioned in the previous points. Namely: 
  \begin{itemize}
  \item How much data plane events can such a middleware handle per second; 
  \item What is the latency penalty for processing such events; 
  \end{itemize}
\end{itemize}

We note also that this thesis has derived from the work presented in the following paper: 

\begin{itemize}
\item Fábio Botelho, Fernando Ramos, Diego Kreutz and Alysson Bessani; \emph{On the feasibility of a consistent and fault-tolerant data store for SDNs}, in Second European Workshop on Software Defined Networks, Berlin, October 2013. 
\end{itemize}

\section{Planning}
The intially proposed work plan was not followed for multiple reasons including the submission and presentation of paper (see previous section), and several misunderstandings regarding the existent \gls{sdn} work that required us us to stop and re-evaluate our understanding of the field. Ultimately this thesis was delayed since it was in the author interest to do so (for private and particular reasons). The total delay accounts  for 3 extra months for the initial prevision of finishing in September  2013. 

\section{Thesis Organization}

This document is structured as following: 
\begin{itemize}
\item Chapter \ref{chapter:1} –-- is the present chapter covering the Introduction;
\item Chapter \ref{chapter:2} –-- covers the background and related work. Namely an overview of \gls{sdn} and state machine replication; 
\item Chapter \ref{chapter:3} --- covers the design of a distributed control plane based on strongly consistent data store; 
\item Chapter \ref{chapter:4} --- covers an evaluation of the data store when integrated with \gls{sdn} applications; 
\item Chapter \ref{chapter:5} --- summaries, and discusses our work.  
\end{itemize}

%%% Local Variables: 
%%% mode: latex
%%% TeX-master: "../PEI"
%%% End: 

\LIMPA

\chapter{Related Work}
\glsresetall

Software Defined Networks (SDN) change the traditional model of network
architecture and design. Traditionally,  networking  relied and evolved over a non-transparent distributed model for
the deployment  of  protocols and configuration of devices. This model
lead to complex protocols and management functions. SDN presents a new way of thinking in networking,
shifting the complexity of protocols and management functions from  the  networks
devices to a general purpose logically centralized service. 

Historically  the network academic environment has followed an 
ad-hoc approach to networking where protocols are introduced as a
response to specific problems. Scoot
Shenker, a networking researcher from Berkley\footnote{Scoot Shenker has played a fundamental role in SDN development, collaborating in many papers. He is also co-founder of both ONF \cite{onf} and Nicira Networks - both very important to SDN.}, ironically sums up  this contribution to networking as a ``big bag of
protocols'' \cite{Shenker:2011ys}. In his opinion there is a lack of
\emph{control abstractions} in current network architectures that, allied  with  their  distributed nature,
lead to the  complex infrastructure available today. Additionally the
network field has failed to developed the appropriate tools for
managing such complexity. 

SDN breaks this complex model through the
physical detachment of the control and data planes.  The motivation behind this
decoupling is the following: if the distributed network state can be collected
and presented to a logically centralized service then it is simpler to both to
specify network level objectives as well as to translate these
objectives into the appropriate configuration of network devices. These
planes, when loosely coupled, can simplify the development of protocols
and the management of network infrastructures. 

In this section we present a historical perspective
of contributions leading to the standardization process of SDN and a
terminology/concept section necessary for
proper understanding of the remaining discussion in this document.  


\section{Software Defined Networks}
\glsresetall
\label{sec:background:sdn}
Software Defined Networking is a networking research field born from a  distributed
collaborative effort. In this section we
outline some of the major contributions that have
led to its current state. 

\subsection{Fundamentals} 
\subsubsection{4D.} The origins of SDN can be traced back to the 4D architecture published in  2005 that ``generated both broad consensus and wide disagreements from
the reviewers''  \cite{Greenberg:2005boa}. In this seminal work the authors identify essential
problems governing  the traditional network architecture and  present an
explanation of the design process behind a clean-slate architectural
pattern for  the development of new networks. 

The authors identify  the root cause of the complexity inherent to the
classic Internet design as the  entanglement  of control and
forwarding functions in the network devices. They advocate that this complexity is tied to the lack of accessible control abstractions in the network devices. 

 With this in mind they present the  4D architecture. Their  design is led by the
decoupling of the control plane from the data plane. Additionally, they
identify three key requirements for the 4D architecture: specification of  network level goals (e.g., routing,
security policy);  centralization of the network state (e.g.,
topology) and finally; direct control of the underlying network
devices.

The 4D architecture can be seen
in Figure \ref{fig:4d}. A description of the four layers
follows: 

\begin{figure}
  \centering 
  \footnotesize
  \includegraphics[scale=0.5]{pic/4d.png}
  \caption[The 4D architecture]{\textbf{The 4D architecture}: is
    composed by 4 planes. Notice that Discovery operates
  in the Dissemination plane.}
  \label{fig:4d}
\end{figure}

\begin{itemize}
\item[] \textbf{Decision:}  defines network-level goals and 
  translates them to configuration primitives of the network
  equipment. This plane maintains a representation of the current state of the network (i.e., the network view); 
\item[] \textbf{Dissemination:}  connects the Decision and Data plane. Must be
  done with robustness guarantees; 
\item[] \textbf{Discovery:} provides discovery information (e.g., network
  equipment, interfaces, etc.) from the Data plane to the Decision plane; 
\item[] \textbf{Data:}  is governed by the network equipment
  infrastructure. Provides packet forwarding functions and implements
  an interface to the Decision plane;  
\end{itemize}

4D effectively removes network logic  from the Data plane
responsibilities. It advocates that what was previously done through complex
routing protocols and per-box configurations (e.g., security policy
definitions)  can be specified in the Decision plane. 

The 4D architecture, even without a concrete implementation, 
presented a significant mark in the history of SDN by setting the
ground for discussion and evolution focused on this essential
decoupling of control and data planes.  

\subsubsection{Ethane.}Ethane \cite{Casado:2007kb} was published in 2007 and presents a
network architecture for the enterprise with emphasis in
security. This work adapts the 4D design to implement a centralized
high-level security policy manager in the controller allied with a namespace  for users and nodes
available in the Network Information base (NIB). The Decision plane is also responsible for
authentication of both users and nodes. 

In Ethane the 4D Decision plane  is instantiated in a complex software
running in a server named \emph{controller}. The controller guarantees
that the security policy is respected. To do so, it can manipulate the network
devices available in the network. The devices perform  flow-based
forwarding  with the help of a local  flow table that is maintained by the
controller. Every flow that fails to match with any of the rules
available in the flow table  is redirected to the controller. After the
control logic is applied the controller may perform several actions
including: forwarding/flooding the packet;  installing flows in the
switch;  drop the packet; etc., The controller plays the interposition role
between any two pair of hosts. The security policy is defined over
high-level names. As
authentication is performed in the controller it is easy to maintain
bindings between these names and dynamic addresses. 

In the paper several replication techniques are presented 
as possibilities for ensuring fault tolerance of the Ethane controller. These are the following: 

\begin{itemize}
\item[] \textbf{Cold standby:}  In this mode backup controllers are passive entities
  waiting to take over if  the active controller fails. Backup
  controllers participate in the
  Minimum Spanning Tree (MST) protocol where the active controller  is
  defined as the root of the tree. Upon failure of the active
  controller the tree converges to a new root (i.e., a new
  controller). Replication only covers authentication and
  network security policy state with simple (unspecified) consistency
  properties that are not strong enough to avoid requiring 
  re-authentication of a subset of  the users when the active
  controller fails. 
\item[] \textbf{Warm-Standby:} In this case  passive controllers recovery
  faster as separate  MST's are maintained for every controller. In this replication
  method the authors also replicate bindings across
  Controllers. However, 
  these are replicated with weak consistency and as so, may require
  re-authentication from users or nodes upon failure of the active controller.

\item[] \textbf{Fully-Replicated}: In this approach two or more active
  controllers are fully replicated. The authors advocate the use 
  of weak-consistent methods based on gossiping but also leave 
  the study open for the use of a fully replicated state machine.
\end{itemize}

Ethane major contribution comes from the practical  experience with
the deployment in the Stanford
University  campus network. The  analysis performed suggests that a single
desktop computer alone could handle a network with over 20K hosts. 

\subsubsection{OpenFlow.} Both Ethane \cite{Casado:2007kb} and 4D \cite{Greenberg:2005boa}
focused in the decoupling of the control plane from the data plane, however there is also required that the control plane can programatically define the data plane configuration. In Ethane this
feature was presented in a monolithic  implementation of the controller.
For general development of SDN  it is  necessary that a
standard interface is available. OpenFlow \cite{openflow}, published
in 2008,  was the first protocol enabling this interface by allowing
the remote manipulation of flow based forwarding network devices. 

OpenFlow (OF) is introduced in the context of an architecture similar to
Ethane \cite{Casado:2007kb}, in which network devices 
are also simple flow-based forwarding equipments. A flow
table resides in the network device and is composed of tuples
$\langle match,action \rangle$. The $match$ entry allows the device to match
arriving packets, while  $action$
specifies the forwarding behavior. Matches can be performed against
standard fields in Ethernet, IP and transport headers while actions can
range from dropping packets; forwarding to single port(s); and/or forwarding to controller. As in Ethane, a non-matching flow is forwarded to the controller who in
turn should instruct  the network device on future behavior through
the modification of the device flow table. Once the table is instructed, 
subsequent packet with matching headers can be forwarded without
the controller interaction. 

Currently the OpenFlow specification is maintained by the Open Network
Foundation (ONF). At the time of writing, the version 1.3 \cite{of13} is available with
several improvements over the original paper \cite{openflow}. 

\subsubsection{Network Operating System.} The Network Operating System (NOS) terminology  was initially used in the
context of Operating Systems (OS's) to name OS's with network services available. In the
context of SDN it was, to our knowledge,
introduced by Gude et al. \cite{Gude:2008jd} and
Cai et al. \cite{Z.-Cai:2008fk}. 

The role
of the Network Operating System is to provide
applications residing above it the ability to effectively control and
observe the state of the network. This ability is provided through a
programmatic interface defined in the Network Operating System itself and should be general
enough to support a broad spectrum of  network management
applications. With the help of this interface it is easier to define
and deploy network control applications such as firewalls and load
balancers. 

This functionality of a NOS is similar to conventional  Operating
Systems (OS's) with the fundamental difference of the managed resources. A
regular OS manages hardware devices from a computer 
and  a NOS manages a network.
The latter environment is harder to manage than the former. 
Aside this difference, the role of the NOS is to implement the
device drivers to communicate with the
network devices and also provide a platform for deployment of network
control applications with integration, communication and
isolation features (as a regular OS does for computational systems). 

In essence, this concept set a clear line in the separation of network applications
from configuration of network devices. The NOS abstraction brings different attributes to the development process of
network control functionality that, until now, was bound by the hardware and respective
software development cycle. Development is still bounded due to the
capabilities of the devices under management of the NOS but at least
the NOS itself and the applications can follow a software-only
development cycle. Additionally one can expect most of SDN functionality to be
heavily dependent   on the applications build on top of NOS much like
current  OS user level functionality is offered by applications
running outside the kernel.

%TODO - they can be a challenge to distributed controllers. 

\subsection{General Architecture}
In March 2011 the Open Network Foundation was created with the
participation and support from several industry partners \cite{onf}. ONF  is a ``non-profit consortium
dedicated to the transformation of networking through the development
and standardization of a unique architecture called Software-Defined
Networking (SDN)''. They have done so, by releasing a white-paper
defining the design principles for a SDN architecture and the
advantages of it \cite{ONF:2012ui}. They are also responsible for the standardization
process of the OpenFlow protocol.

It is important to emphasize that Software Defined Networks is not a
standard. It should be
clear, by now, that SDN's are an architectural pattern with two
essential properties:

\begin{itemize*}
\item Decoupling of the control plane from the data plane;
\item The network control must be driven by software.
\end{itemize*}

In this section we present the SDN architecture used as reference throughout this document.  

\subsubsection{General architecture.} The architecture of Software Defined Network is presented in Figure
\ref{fig:sdn-stack}. A bottom up explanation of the responsibilities of
each layer follows:

\begin{figure}
  \centering 
  \footnotesize
  \includegraphics[scale=0.5]{pic/sdn-stack.png}
  \caption[SDN Architecture]{\textbf{SDN Architecture}: is composed by three
    layers: in \textbf{Management} resides the application
    logic that governs the overall network behavior; The
    \textbf{Network Operating System} (NOS)  allows the integration
    of  Management applications and exposes an interface for the
    manipulation of the network (northbound API). It also exposes the Network
    Information Based (NIB); Finally the network is
    represented in the \textbf{Infrastructure} layer. The devices must
  implement an interface allowing configuration (southbound API)}
  \label{fig:sdn-stack}
\end{figure}

\begin{itemize}
\item[] \textbf{Infrastructure Layer:} In this layer are the network
  devices responsible for packet forwarding. Any device 
  can be used (wireless access point, Ethernet switch, router) as long as it implements 
  a standard configuration interface (e.g., OpenFlow). Throughout the text we refer to all these devices as switches;
\item[] \textbf{Network Operating System (NOS):} Provides a standard
  interface to the upper layer (i.e., the northbound interface) allowing the manipulation of the network
  state such as forwarding tables in the managed devices. The
  configuration of devices is actually done by the NOS by interaction
  with the Infrastructure interface for configuration (i.e., the southbound interface). Additionally, it should provide features
  for the integration of Management layer applications.  Throughout
  the text we use NOS, controller or control plane to refer to this
  layer;
\item[] \textbf{Management:} This is where the network logic operates. In this layer
  resides the definition of network level objectives in the form of
  one or more applications. These applications interact with the NIB
  to consult and modify the network state. 
\end{itemize}

The NOS layer provides the northbound API to the upper layers. Network
applications run in the Management layer and can interact with the
network through this API. The NOS layer is also responsible for
communicating with the network devices through the southbound API. The
usual implementation of this API is the OpenFlow protocol.

Notice that the Network Operating System layer plays the intermediary
role in the stack. This implies that not only it allows that
Management applications alter the network state but  also has to
communicate to the Management layer the events that occur at the
Infrastructure level. Complex controllers such as Onix
\cite{Koponen:2010th} use the NIB as the intermediary in this duplex
communication while  the vast majority of controllers only use the NIB as a
simple datastore. 

There must be reliable connectivity between the Network Operating
System and the Infrastructure. The connectivity may be 
\textbf{in-bound} or \textbf{out-bound}
. In the in-bound case the connectivity takes place over the
network used for data forwarding while in the out-bound case a
different and isolated network is used. Connectivity
between these two layers require manual configuration of the
Infrastructure components. 

\subsubsection{Types of Controllers.} The controller is often
categorized in the literature
``logically centralized'' \cite{Gude:2008jd,Greenberg:2005boa}. This concept is used in distributed systems literature
to refer to a physically distributed system that appears, to
its users,  as a
coherent and  transparently distributed service (i.e., it does not
appears to be distributed). The term is perhaps
not well employed in SDN literature as pointed out in
a blog post, by Martin Casado, a Stanford researcher and Nicira
co-founder \cite{:zr}. We will not maintain
this terminology either. Instead, we define that the control platform is 
either  \textbf{distributed} or 
\textbf{centralized} for the remaining discussion. In the centralized
case a single controller is responsible for all the network
Infrastructure as opposed to the distributed case where several
controllers are used.

Another category  distinction in controller software is based on another text
from Casado and Koponen \cite{Martin-Casado:2011ly}, where
three categories of controllers are discussed: 

\begin{itemize}
\item[] \textbf{Single Purpose Controller:} lack support for general management
  applications. Ethane is an example of this type of controllers; 
\item[] \textbf{Thin Controller:} present a northbound interface that is
  strongly-coupled to the southbound interface. Most controllers fall
  under this category. Usually these controllers are known as OpenFlow controllers given the use of OpenFlow in the southbound interface.  
\item[] \textbf{General Controllers:} offer a general purpose service with loosely
  coupled south and northbound interfaces. Transition from the OpenFlow protocol for
  other protocol may  be completely transparent to the Management layer.
\end{itemize}



\section{Physically Centralized  Controllers}
\glsresetall
\label{sec:background:centralized}


In this section we present an overview of relevant centralized
controllers. Centralized controllers are, by our definition, control
planes that do not show any explicit support in the deployment of a
different controllers processes over onde or more servers. 
\subsection{Existent Controllers}
%TODO - what defines a centralized controller. 
%TODO - pipeline as a common model
%TODO - event based
%TODO - thin. 
\subsubsection{NOX}
\label{sec:nox}
NOX \cite{Gude:2008jd}  was published and publicly released under
GPL in 2008. It was both developed in C++ and Python. NOX enables a standard interface for the integration of  Management applications 
in the controller. These applications control
network objectives and also cooperate to define the current
network view (i.e., the NIB). The view is shared between applications. One of the
contributions of the article was the definition of the Network
Operating System abstraction for the controller service. However, NOX
is a Thin Controller. 

\begin{figure}
  \centering 
  \footnotesize
  \includegraphics[scale=0.5]{pic/nox-pipeline}
  \caption[NOX pipeline] {\textbf{NOX pipeline} An overview of the NOX
    pipeline used for event processing by the Management applications. NOX
  receives events that have originated either in the Infrastructure of
the Management layer and dispatches them through a pipeline of
applications who have registered for processing these events. 
 As an example: \emph{PACKET\_IN} is a network event while
\emph{USER\_AUTHENTICATED} could be an application event.}
  \label{fig:nox-pipeline}
\end{figure}

NOX is a component framework with primitives for
construction and deconstruction of OpenFlow
based messages. The programming model is event-based. Applications (the components) are registered in a
priority based pipeline with event handlers associated to either
OpenFlow or applications events. This process can be seen in Figure
\ref{fig:nox-pipeline}, which describes the NOX dispatching behavior. Notice
that applications decide if the event should continue to be processed
by the pipeline. NOX  currently ships
with several applications (e.g., forwarding, topology discovery, host
tracking, spanning tree, layer two switch behavior, etc.).

NOX is a centralized controller  but the authors argue that it can easily be distributed for resilience if the shared state (the \emph{view}) is consistently distributed. 
Initially it was a single threaded application not focused on
performance. However, from its publishing date 
several improvements have taken place
\cite{Tootoonchian:2012uia,zen-doc-thesis} that have significantly improved
NOX performance. Under the set of improvements we highlight the
natural evolution to a multi-core aware application
that statically distributes network requests to different threads. 

In the time of writing, NOX is publicly available but has ramified into
two different applications: A C++ based controller available in
Linux and a Python  based controller (POX) available for
several environments \cite{nox}.

\subsubsection{Maestro}

Maestro is the undergoing work of Zeng Cai covered in
\cite{maestro}. It is presented as a Network Operating System focused
in coordination and isolation of the applications  that control the
infrastructure layer. Cai recognizes that Management components do not
operate independently and in isolation. Instead, they operate
concurrently with inter-dependent state (present in the NIB). With this in mind
it aims to exploit parallell computing benefits in the control plane. 

Maestro splits the regular pipeline execution such that it can
be concurrently executed. As seen in Figure \ref{fig:maestro-pipeline} events may
follow different execution paths since singular control components are
not interested in every single event. Thus, Maestro can manage to
execute several applications concurrently. However, in order to
coordinate the control component access to the NIB Maestro opts to
have a more granular network state model. The author argues that it is
common to control components to be  interested only in  subsets of the
NIB. In order to employ concurrent execution Maestro requires that
applications specify  what subsets of the NIB they require as input
and what subsets they modify as output. 

Maestro employs
coordination of the  execution of the applications with performance in
mind. As an example based on Figure \ref{fig:maestro-pipeline} if the routing table
(subset of the NIB) is updated while the RouteFlow application is running, then
Maestro makes sure that the application will use the old version of
the RoutingTable.

\begin{figure}
  \centering 
  \footnotesize
  \includegraphics[scale=0.5]{pic/maestro-pipeline.png}
  \caption[Maestro pipeline]{\textbf{Maestro pipeline} Maestro split the pipeline execution
  into several concurrent pipelines based on the events applications
  process and the state they access or modify. In the figure we can
  see two execution paths: \textbf{Topology Changes} processes the
  events triggered by changes in the Infrastructure and \textbf{Flow
    Requests} processes new flow events.}
  \label{fig:maestro-pipeline}
\end{figure}


The fundamental objective of Maestro is to maximize scalability in a
centralized control plane. To do so it attempts to exploit
parallelization in the controller server. Three major design goals
shape Maestro: fair distribution of work across cores; minimal
overhead introduced by cross-core and cache synchronization and;
minimal memory consumption. In addition, it also
exploits throughput optimization through batching. The results
published show that Maestro linearly scales the throughput with the number of cores
available on the controller. 


Currently Maestro is available under the LGPL 2.1 licence. It ships
with usual switching  and routing capabilities \cite{maestro}.

\subsubsection{Beacon}
\label{sec:beacon}
Beacon is an open source controller built in Java, by David Erickson during his academic studies in Stanford University. 
He is, to our knowledge, the only official maintainer of the
application. 

Beacon is also a Thin Controller with  an event-based programming model. Applications register for
specific type of events and process these  in the order
configured by the user. Any application processing an event chooses to forward the
event further in the pipeline or terminate its execution. It is also
multi-threaded, binding switches to particular threads. Applications receive data from all threads.

Applications in Beacon are implemented as \emph{bundles}. A bundle is the
unit of abstraction in the OSGI \cite{osgi} framework - a component and service
platform for the Java programming language with dynamic capabilities -
allowing features such as \emph{hot-swapping} (i.e., deploy, start and
stop modules in run time). 
Beacon provides a central service (the registry) for registration of bundles as
services. Each bundle implements a service, exports it to the registry and
other bundles may consume it. Applications events in Beacon take place
through the service abstraction: bundles may register in other bundles as
listeners to be notified when for specific events take place. 
%TODO - reescrever essa ultima frase. 

Beacon does not provide any NIB service. The network state is
decentralized and encapsulated in the bundle abstraction. There are
no persistance  mechanisms also. 

At the time of writing the applications available are the following: 
 learning switch, hub, device manager , topology, layer 2
shortest path routing, arp  proxy, dhcp proxy. 

%\cite{Controllers: Beacon with David Erickson. Open Network Summit
%  http://www.youtube.com/watch?v=tZ3G_FDuMjg}

\subsubsection{Floodlight}
Floodlight is an open source Apache licensed controller. It was
initially forked from Beacon. It  is developed and maintained by an open community of developers that is mainly composed of Big Switch\footnote{A SDN vendor with a commercial
distributed controller named Big Controller \cite{:vn}.} employers. It is written in Java, but applications can either be
implemented in Java, Jython or through the REST service
available in the NOS (with limited functionality). Floodlight is also a Thin Controller. 

Floodlight follows the common event driven
programming model of most  controllers. Although Floodlight was originally
forked from Beacon, the OSGI support was taken for performance and
deployment reasons. The overall functionality is based on modules
(i.e., applications) that implement services that can be consumed by
other modules. It is similar to Beacon in this regard, however the
module/service functionality is directly provided by Floodlight
instead of delegated to a third-party framework as OSGI. 

Floodlight is also multi-threaded. It accomplishes this through an
asynchronous event based multithreaded library named Netty \cite{netty} that manages Input/Ouput communication with the managed
switches. 

Currently the applications available are: topology manager,  link
discovery, forwarding, device manager, storage, firewall and
static flow pusher.  

\subsection{Controller Choice}
In this section we present an overview of relevant centralized
controllers. Centralized controllers are, by our definition, control
planes that do not show any explicit support in the deployment of a
different controllers processes over onde or more servers. 

%TODO - what defines a centralized controller. 
%TODO - pipeline as a common model
%TODO - event based
%TODO - thin. 
\subsection{NOX}
\label{sec:nox}
NOX \cite{Gude:2008jd}  was published and publicly released under
GPL in 2008. It was both developed in C++ and Python. NOX enables a standard interface for the integration of  Management applications 
in the controller. These applications control
network objectives and also cooperate to define the current
network view (i.e., the NIB). The view is shared between applications. One of the
contributions of the article was the definition of the Network
Operating System abstraction for the controller service. However, NOX
is a Thin Controller. 

\begin{figure}
  \centering 
  \footnotesize
  \includegraphics[scale=0.5]{pic/nox-pipeline}
  \caption[NOX pipeline] {\textbf{NOX pipeline} An overview of the NOX
    pipeline used for event processing by the Management applications. NOX
  receives events that have originated either in the Infrastructure of
the Management layer and dispatches them through a pipeline of
applications who have registered for processing these events. 
 As an example: \emph{PACKET\_IN} is a network event while
\emph{USER\_AUTHENTICATED} could be an application event.}
  \label{fig:nox-pipeline}
\end{figure}

NOX is a component framework with primitives for
construction and deconstruction of OpenFlow
based messages. The programming model is event-based. Applications (the components) are registered in a
priority based pipeline with event handlers associated to either
OpenFlow or applications events. This process can be seen in Figure
\ref{fig:nox-pipeline}, which describes the NOX dispatching behavior. Notice
that applications decide if the event should continue to be processed
by the pipeline. NOX  currently ships
with several applications (e.g., forwarding, topology discovery, host
tracking, spanning tree, layer two switch behavior, etc.).

NOX is a centralized controller  but the authors argue that it can easily be distributed for resilience if the shared state (the \emph{view}) is consistently distributed. 
Initially it was a single threaded application not focused on
performance. However, from its publishing date 
several improvements have taken place
\cite{Tootoonchian:2012uia,zen-doc-thesis} that have significantly improved
NOX performance. Under the set of improvements we highlight the
natural evolution to a multi-core aware application
that statically distributes network requests to different threads. 

In the time of writing, NOX is publicly available but has ramified into
two different applications: A C++ based controller available in
Linux and a Python  based controller (POX) available for
several environments \cite{nox}.

\subsection{Maestro}

Maestro is the undergoing work of Zeng Cai covered in
\cite{maestro}. It is presented as a Network Operating System focused
in coordination and isolation of the applications  that control the
infrastructure layer. Cai recognizes that Management components do not
operate independently and in isolation. Instead, they operate
concurrently with inter-dependent state (present in the NIB). With this in mind
it aims to exploit parallell computing benefits in the control plane. 

Maestro splits the regular pipeline execution such that it can
be concurrently executed. As seen in Figure \ref{fig:maestro-pipeline} events may
follow different execution paths since singular control components are
not interested in every single event. Thus, Maestro can manage to
execute several applications concurrently. However, in order to
coordinate the control component access to the NIB Maestro opts to
have a more granular network state model. The author argues that it is
common to control components to be  interested only in  subsets of the
NIB. In order to employ concurrent execution Maestro requires that
applications specify  what subsets of the NIB they require as input
and what subsets they modify as output. 

Maestro employs
coordination of the  execution of the applications with performance in
mind. As an example based on Figure \ref{fig:maestro-pipeline} if the routing table
(subset of the NIB) is updated while the RouteFlow application is running, then
Maestro makes sure that the application will use the old version of
the RoutingTable.

\begin{figure}
  \centering 
  \footnotesize
  \includegraphics[scale=0.5]{pic/maestro-pipeline.png}
  \caption[Maestro pipeline]{\textbf{Maestro pipeline} Maestro split the pipeline execution
  into several concurrent pipelines based on the events applications
  process and the state they access or modify. In the figure we can
  see two execution paths: \textbf{Topology Changes} processes the
  events triggered by changes in the Infrastructure and \textbf{Flow
    Requests} processes new flow events.}
  \label{fig:maestro-pipeline}
\end{figure}


The fundamental objective of Maestro is to maximize scalability in a
centralized control plane. To do so it attempts to exploit
parallelization in the controller server. Three major design goals
shape Maestro: fair distribution of work across cores; minimal
overhead introduced by cross-core and cache synchronization and;
minimal memory consumption. In addition, it also
exploits throughput optimization through batching. The results
published show that Maestro linearly scales the throughput with the number of cores
available on the controller. 


Currently Maestro is available under the LGPL 2.1 licence. It ships
with usual switching  and routing capabilities \cite{maestro}.

\subsection{Beacon}
\label{sec:beacon}
Beacon is an open source controller built in Java, by David Erickson during his academic studies in Stanford University. 
He is, to our knowledge, the only official maintainer of the
application. 

Beacon is also a Thin Controller with  an event-based programming model. Applications register for
specific type of events and process these  in the order
configured by the user. Any application processing an event chooses to forward the
event further in the pipeline or terminate its execution. It is also
multi-threaded, binding switches to particular threads. Applications receive data from all threads.

Applications in Beacon are implemented as \emph{bundles}. A bundle is the
unit of abstraction in the OSGI \cite{osgi} framework - a component and service
platform for the Java programming language with dynamic capabilities -
allowing features such as \emph{hot-swapping} (i.e., deploy, start and
stop modules in run time). 
Beacon provides a central service (the registry) for registration of bundles as
services. Each bundle implements a service, exports it to the registry and
other bundles may consume it. Applications events in Beacon take place
through the service abstraction: bundles may register in other bundles as
listeners to be notified when for specific events take place. 
%TODO - reescrever essa ultima frase. 

Beacon does not provide any NIB service. The network state is
decentralized and encapsulated in the bundle abstraction. There are
no persistance  mechanisms also. 

At the time of writing the applications available are the following: 
 learning switch, hub, device manager , topology, layer 2
shortest path routing, arp  proxy, dhcp proxy. 

%\cite{Controllers: Beacon with David Erickson. Open Network Summit
%  http://www.youtube.com/watch?v=tZ3G_FDuMjg}

\subsection{Floodlight}
Floodlight is an open source Apache licensed controller. It was
initially forked from Beacon. It  is developed and maintained by an open community of developers that is mainly composed of Big Switch\footnote{A SDN vendor with a commercial
distributed controller named Big Controller \cite{:vn}.} employers. It is written in Java, but applications can either be
implemented in Java, Jython or through the REST service
available in the NOS (with limited functionality). Floodlight is also a Thin Controller. 

Floodlight follows the common event driven
programming model of most  controllers. Although Floodlight was originally
forked from Beacon, the OSGI support was taken for performance and
deployment reasons. The overall functionality is based on modules
(i.e., applications) that implement services that can be consumed by
other modules. It is similar to Beacon in this regard, however the
module/service functionality is directly provided by Floodlight
instead of delegated to a third-party framework as OSGI. 

Floodlight is also multi-threaded. It accomplishes this through an
asynchronous event based multithreaded library named Netty \cite{netty} that manages Input/Ouput communication with the managed
switches. 

Currently the applications available are: topology manager,  link
discovery, forwarding, device manager, storage, firewall and
static flow pusher.  


\section{Distributed Controllers}
\glsresetall
\label{sec:relatedWork:distributed}

In this section we provide an overview of Distributed Controllers
existent in the literature. By our definition an distributed
controllers provides explicit support to the deployment of several
controller processes over one or more servers. 

\subsection{HyperFlow}
Motivated by the lack of scalability in centralized controllers, HyperFlow \cite{Tootoonchian:2010vy} 
was created as a distributed controller. The 
authors aim was to provide scalability without sacrificing
the simplicity of management applications  in  centralized controllers. It
is built as a \emph{C++} application on top of the NOX 
controller \cite{Gude:2008jd} requiring minor modifications to the
controller core and is not publicly available. 

\subsubsection{Architecture.} HyperFlow is composed of two main components: the controller application
and an event propagation system. The overall architecture can be seen
in figure \ref{fig:hyperflow-design}. The controller component is
an application built on top of NOX \cite{Gude:2008jd} which implements an event logger
responsible for subscribing to OpenFlow events and a command proxy
capable of sending OpenFlow commands to  network devices. In essence
HyperFlow interposes acts as an intermediate layer between NOX and
Management applications. The event
propagation system allows  communication between HyperFlow
instances and  is built over a distributed filesystem. Communication
is done through channels implemented as files. There are three types of channels: the
control channel where controllers advertise themselves; the data
channel where general interest events are published by every
controller;  and finally the individual controller channel for OpenFlow
commands relevant to the controller who owns the channel. The
communication system is based on the well known \emph{Publish/Subscribe}
model  whereby one defines publishers as senders of
messages and subscribers as the receivers. Publishers do not send
messages directly  to receivers.  Instead,
messages are published in a medium (in this case the channels) and
interested receivers subscribe to publishers through some form of
subscription logic (e.g., channel, topics, etc.). The Publish/Subscribe
model allows the decoupling of both space and time in the
communication between publishers and subscribers. 
HyperFlow requires that the communication system  provides:
persistent storage of the events published; \emph{FIFO}
(First-In-First-Out)  ordered delivery of
the messages published by the same controller; and resilience against
network partitioning.

\begin{figure}
  \centering 
  \footnotesize
  \includegraphics[scale=0.5]{pic/hyperflow-design.png}
  \caption[HyperFlow architecture]{\textbf{HyperFlow architecture} is  similar to
    NOX  with the addition of an Publish/Subscribe middleware
    used between the HyperFlow Management application. The
    Publish/Subscribe system is used for all communication between controllers.}
  \label{fig:hyperflow-design}
\end{figure}


\subsubsection{State Distribution.} State distribution in HyperFlow is accomplished through the
Publish/Subscribe event propagation system. The NIB  is
replicated in all instances and HyperFlow distributes the events
that cause changes to it. This is done through event publishing in the shared
data channel. Every controller subscribes to the
data channel and replays the events received on it, thus changing its
NIB in a similar way.  The view  is maintained by the Management applications
residing in the controller and outside the domain of
HyperFlow (as in NOX). Although not specified in the article HyperFlow attempts to
behave in a similar form to a distributed state machine. However it
lacks total order and does not specify that applications should be
replicated and deterministic. 

HyperFlow does not guarantee strong
consistency as events are not totally ordered  between controllers. 
Notice that even with FIFO based channels some controller $c$ might receive event $e_i$ followed by event
$e_j$ sent from controllers $i$ and $j$, while a controller $c'$
perceives $e_j$ followed by $e_i$. It is possible that such occurrence
leads controllers to divergent states. The article explicitly adverts
that Management applications should not rely in event ordering "except
those targeting the same entity (e.g., the same switch or link)'' as
those guarantee FIFO ordering. 

Additionally, Hyperflow  also addresses incorrectness problems caused by
transient inconsistency across controllers  by defining an
\emph{authoritative}  controller for each flow. This controller is
responsible for orchestrating changes in the network regarding some
flow. As an example, to avoid loop-free forwarding \emph{authoritative}
controllers are solely  responsible for setting the flow paths across the forwarding
plane for some specific flow. For this, applications must relay
requests for some flow to its \emph{authoritative} controller. 

\subsubsection{Scalability.} HyperFlow addresses scalability of the control plane by minimizing the
number of events that an instance replicates to others. Thus it is
focused on the scalability of the CPU and does not address state
scalability since the NIB is fully replicated in every controller. 

To minimize the number of events  processed by
instances, HyperFlow filters the dissemination of events. To this end
it requires that applications tag locally generated events that affect
the NIB. Only these events are worth distributing as others are redundant. 

Local events are generated either by the Management or
Infrastructure layers but Management applications trigger events as a
response to the processing of other events (i.e., there is a causality
relationship between events). Thus, HyperFlow
minimizes distribution of events even further and if, for example,
some event $e$ is triggered due to the processing event $i$, it 
distributes only event $i$. To this end, Management applications
triggering events have to, both flag the event if it affects the NIB state
but also to associate the event that caused it. As one single event
can cause several application events to be triggered this can reduce
the volume of traffic and processing of redundant events.

\subsubsection{Programming Model.} Applications running on HyperFlow act as they control the entire
network. Behind the scenes  HyperFlow redirects requests to the
network equipment  either to the \emph{authoritative} controller or to the
controller managing the equipment which the request addresses. The choice depends on the type of
the request. HyperFlow also routes back  responses to the controller
responsible for  the request. Also, as previously said, every state
changing event is seen by every controller. So in the end the NIB of
the network contains updated information from every  equipment present
in the network.

The programming model itself is identical to NOX  (event driven,
pipeline based) thus
maintains its simplicity. Some overhead complexity is added as
HyperFlow requires the applications to tag  state changing events and
identify parent events as explained above. 

\subsection{Onix}
Onix \cite{Koponen:2010th}  was build upon the NOX legacy and
provides two major contributions: it is the first general controller
published in the literature, and also the first to provide strong consistency.  Onix 
provides an improved Network Operating System interface on 
which the northbound API does not reflect the southbound
API. Management applications are programmed against a network
graph of Typed Entities very similar to the Object Oriented paradigm
and are not aware of the southbound characteristics (e.g., the use of Openflow). The graph is implemented in 
the \emph{Network Information Base}
(NIB) and can be distributed across a  cluster of  Onix
controllers. Applications have the choice to specify consistency and
durability requirements  per network entity  present in the NIB. 
Onix was the result of a joint effort between Google, NEC, Nicira,
ISCI and Berkely, and (at least)  both Google and Ericson have
developed their controllers based on from Onix \cite{The-Valley-of-the-Nerd.:fk}. At the time of
writing Onix is not publicly available. 

\subsubsection{Architecture.} Onix architecture is presented in figure \ref{fig:onix-design}. Only
one application resides on the 
Management layer and may communicate with other applications in other
controllers instances for coordination. The NIB is the
only element in Onix northbound interface. The Management layer
directly modifies the NIB and subscribes to changes on it. The Infrastructure layer
indirectly modifies the NIB (trough the NOS layer). The NOS has to
guarantee that changes in the Infrastructure are reflected in the NIB
and vice versa. For this, it translates network events into changes in
the NIB and changes in the NIB to changes in the Infrastructure configuration. Onix
supports This process is represented in figure
\ref{fig:onix-process}.  OpenFlow but it could transparently move to another
southbound API. 

Each Onix instance  independently manages a subset of the
Infrastructure. However the NIB  exposes all the network for each
instance. As such, each time a local Onix instance alters the NIB these
changes are reflected in all other instances. Thus, the NIB is also
the distribution mechanism of the Onix controller. The  distribution
itself is done by the datastores that back up  the NIB state (seen
in figure \ref{fig:onix-design}).

\begin{figure}
  \centering 
  \includegraphics[scale=0.5]{pic/onix-design.png}
  \caption[Onix architecture] {\textbf{Onix achitecture.} In Onix the NIB is the
    solely entity used as the northbound api and Managements program
    directly against it. The NIB is supported by
    two replicated datastores accessible across Onix instances.  The
    Management layer communicates across instances for coordination.}
  \label{fig:onix-design}
\end{figure}

\begin{figure}
  \centering 
  \includegraphics[scale=0.5]{pic/onix-process.png}
  \caption[Onix configuration process]{\textbf{Onix configuration
      process}. The interactions between layers of the Onix
    instance stack. In the figure we can observe the process of
    configuration on the left side, and the process of detecting
    changes in the Infrastructure on the right side.\footnote{The Infrastructure should be only one.}} 
  \label{fig:onix-process}
\end{figure}

\subsubsection{State Distribution.} Onix defines a flexible distribution model for the NIB 
whereby  it offers the application
designer the choice of consistency guarantees. Two
replicated datastores are present, covering  strong 
and  eventually consistency. Strong consistency data is
provided through a transactional persistent database backed up by a replicated
state machine. This datastore is
favored for data with low-frequency changes  as its performance limitations are
significant. The eventually consistent datastore consists in 
an one hop memory based DHT  (as Dynamo 
\cite{DeCandia:2007cn}) favored for  volatile data with high update
rates. 

The NIB reflects the state of both datastores. Both the integration of
datastores and the inconsistency  characteristics of the DHT can lead the
NIB to an inconsistent state where reads performed in some entity may
return more than one result. Thus Onix provides primitives
for the integration of inconsistency resolution logic as well as 
direct integration of a distributed
coordination framework (see  ZooKeeper \cite{Hunt:2010ux}) in the
northbound interface. 

\subsubsection{Scalability.} Onix is intended for large scale Infrastructure where scalability is
fundamental. In each Onix instance the NIB
size reflecting the network state could lead to memory
exhaustion   and the processing of both network events and
subscriptions 
to changes in  the NIB can lead to CPU exhaustion.

In order to introduce scalability and avoid exhaustion, partition and
aggregation based techniques can be configured in the Management
layer. Partition avoids full replication of both data  and workload
such that additional instances do not only replicate overall work but
also relief it. As for aggregation it can, for example, allow network entities to be
aggregated and exposed for other Onix instances as only one.

In practice partition in Onix is present in the division of the
Infrastructure across different Onix instances. 
This way
instances process fewer events. Additionally the Management logic can
configure Onix instances to keep only subsets of the NIB in memory and
up to date. For aggregation, an Onix instance can be configured to
expose a subset of the NIB as an aggregated element. 

\subsubsection{Programming Model.} Applications are built against the NIB graph data structure that is
composed of \emph{Typed Entities} supporting the Object
Oriented paradigm (i.e., encapsulation of data, functions over
entities, hierarchy, etc.). Onix supports extensible representations of
network entities. The
API provides essential functions to search, inspect, create, destroy and
modify entities present in the NIB. It is also possible to register
notifications for creation, removal and updates of data
entities. When network events of other Onix instances update the
datastores those changes must be reflected on the local NIB and the
application must be notified through a callback function. 
All operations are asynchronous, with
eventual delivery and no ordering or latency guarantees given
therefore a \emph{barrier} synchronization primitive is available
allowing the application to wait as
updates are translated and applied in the network devices and/or other
controllers. 

Finally it is worth mentioning that Onix employs coarse-grained
locking mechanisms over the NIB. Applications are given the guarantee
that no other local thread concurrently updates the NIB. 

\subsection{Kandoo}
Kandoo \cite{Yeganeh:2012jm} is an hierarchical controller for
scalable infrastructures. The main
contribution comes from the deployment of isolated controllers near
the switches that shield a parent  controller from processing all
events originating at the network. It is implemented in a mixture of
C, C++ and Python and is not publicly available. 

\subsubsection{Architecture.} In Kandoo --- see Figure  \ref{fig:kandoo-design} --- the controller is split in two levels: in the top level (Network Operating System layer) we
have the root controller responsible for normal operation. In the
bottom (at the Infrastructure layer) we have local controllers that are (ideally)  closer to the
managed switches. This two level design allows  shielding the
controller from frequent events triggered by 
the network devices. Notice that  the entire SDN
stack is replicated in the Infrastructure with the exception of the
NIB since the bottom Management layer should not require direct access to the NIB services. If necessary access to NIB
must be directed to the root controller.

\begin{figure}
  \centering 
  \includegraphics[scale=0.5]{pic/kandoo-design.png}
  \caption[Kandoo design] {\textbf{Kandoo design.} Kandoo decomposes the usual controller
    design into two layers. For this, it brings both controllers and
    applications closer to the Infrastructure. The controllers
    residing in the Infrastructure layer are responsible for
    processing frequent events that do not require access to the
    network state present at the NIB.} 
  \label{fig:kandoo-design}
\end{figure}

This design is motivated by the ideia of bringing control
functionality towards the datapath. As such, \emph{Local Apps} should
require no network wide state and the bottom controllers should stay
close to switches. The deployment of the bottom control plane can
even happen directly at the switches if they support this
functionality (e.g., software switches). 

The remaining design of the controller is not well specified in the
paper. However, other architectures can complement the Kandoo architecture.
The authors stress that even distributing the root
control plane through Onix or HyperFlow is orthogonal to the remaining design.

\subsubsection{Scalability.} In Kandoo scalability is supported by the introduction of the two
level hierarchy. The bottom control plane  shields the root
control from processing of events. As the bottom plane does not
requires access to the NIB it remains simple and
efficient. Additionally, as it is closer to the data plane latency
penalties are lower. However the effectiveness of this
scalability mechanism depends on the Management functionality. For
effective shielding one requires Management applications capable of
operating without access to the network state. The article does not
specifies how practical or significant are this class of
applications beyond exemplifying local policy enforcement  and the
Link Layer Discovery Protocol. 

\subsubsection{Programming Model.} In Kandoo the NOS layer is responsible for the deployment of local
applications in the bottom control plane. The application model only
requires that the applications deployed at the NOS have a flag
that sets its scope (i.e., local or non-local). All the remaining work
is left for the root controller. 

% \subsection{Comparison to Blah, Critic}
% This section is only a scratch for future work orientation. It is
% intended as a comparison to the use of the NIB we are going to build
% and also to criticize the limitations and problems of both HyperFlow
% and Onix. 
% \begin{description}
% \item[TODO] make it clear that HyperFlow replicates controllers. 
% \item[Scalability in Onix] It doesn't exist in the published work (By
%   now it should exist off course). There is quite a bit of blah in the
%   paper about techniques and such but in reality all is left for the
%   application layer to implement relying (again!) on distributed
%   coordination and locking mechanisms; 
% \item[Strong consistency in Onix] Not 100 \% sure about this: it does
%   not exist in practice. If an application mixtures Strong consistency
%   nodes with eventually consistent  links (i.e., (Object A is strong consistent and
%   contains B eventual consistent) it must be prepared to deal
%   with inconsistent resolution upon dealing with the node. The paper
%   suggests this. It is
%   overly complicated in the application logic. 100 \% strong
%   consistency simplifies a lot the application development. 
% \item[Typed entities, Nib Abstraction in Onix] Really cool. higher
%   level of abstraction, another level of indirection. Typed entities
%   combined with the NIB is eternal code loosely coupled to management
%   protocols used in the management of network equipment. Casado
%   defends in a network post
%   [http://networkheresy.com/2011/08/09/what-might-an-sdn-controller-api-look-like-and-should-we-standardize-it/]
%   that it is a general loosely coupled layer, independent of the state
%   distribution mechanism. I disagree, at least in the published
%   work. The state distribution is ``à vista'' in the application layer
%   (define consistency per entity, conflict resolution). It is not
%   fully transparent and consequently not loosely coupled. 

% \item[HyperFlow concept] Kind of reminds a distributed state machine
%   at application level (replay all relevant controller events -i.e.,
%   state changing events) but lacks ordering requirements. 

% \item[HyperFlow] favors availability over consistency. O nosso não
%   fará isso. 
% \item[HyperFlow] sucks. guarantees a bounded window of inconsistency
%   if the network changes trigger less than around 1000 events per
%   seconds. 
% \item[Onix] allows both. The Management deciding. Sacrificing
%   transparency, simplicity. Also, it sucks. 

% \item[Resumo] Hyerflow distribui os eventos o que e estupido porque há
%   muito mais eventos que alteracoes na NIB. Então ele distribui só os
%   eventos que mudam a NIB. O Onix oferece os dois tipos de coerencia
%   mas complica a programacao porque Entities que combinam dois tipos de
%   coerencia podem sempre retornar dois resultados. O Kandoo mete
%   hierarquia ao barulho para minimizar o numero de eventos. 


% \end{description}

\section{Consistent Data Stores}
\glsresetall
\label{sec:relatedWork:consistentDataStore}
%Section data stores do artigo. 
The key idea of our controller architecture is to make the controller instances coordinate their actions through a dependable data store in which all relevant state of the network and of its control applications is maintained in a consistent way.
This data store is implemented with a set of servers (replicas) to avoid any single point of failure, without impairing consistency.
One of the most popular techniques for implementing such replicated data store is state machine replication (SMR)~\cite{Sch90,Lam98}.
In this section we review the state of the art on replicated data stores and describe some reasons why, contrary to common belief, they can be a valid option for supporting a distributed controller architecture.

Practical crash fault-tolerant replicated state machines are usually based on the Paxos agreement algorithm for ensuring that all updates to the data store are applied in the same order in all replicas (thus ensuring consistency)~\cite{Lam98}.
Since the original Paxos describes only an algorithmic framework for maintaining synchronized replicas with minimal assumptions, we instead describe the Viewstamped Replication (VR) protocol, a similar (but more concrete) state machine replication algorithm introduced at the same time~\cite{reitblatt2012abstractions}.
Fig. \ref{fig:paxos} shows the messages exchanged in Paxos/VR for an update operation: the client sends a message to a primary replica (the leader) that disseminates the update to all other replicas. These replicas write the update to their log and send an ACK to the primary.
In the final step the leader executes the request and sends the reply to the client.
If the primary fails, messages will not be ordered and thus a new primary will be elected to ensure the algorithm makes progress.
When read-only operations are invoked, the leader can answer them without contacting the other replicas.
Strong consistency is ensured due to the fact that all requests are serialized by the leader.

\begin{figure}[!ht]
\centering
\includegraphics[width=0.5\textwidth]{pic/related/vr.pdf}
\caption[Paxos/VR update protocol]{Paxos/VR update protocol.} 
\label{fig:paxos} 
\end{figure}

The Paxos/VR algorithm has served as the foundation for many recent replicated (consistent and fault-tolerant) data stores, from main-memory databases with the purpose of implementing coordination and configuration management (e.g., Apache'  Zookeeper~\cite{Hun10}), to experimental block-based data stores or virtual discs~\cite{Rao11,Bol11,Bes13}, and even to wide-area replication systems, such as Google Spanner~\cite{Corbett:2012uz}.
Besides the synchronization protocol, these systems employ many implementation techniques to efficiently use the network and storage media.

Although not as scalable as a weakly consistent data store, these systems grant the advantages of consistency for a large number of applications, namely those with moderate performance and scalability requirements. To give an idea of the performance of these systems, Table \ref{table:smr-results} shows the reported throughput for read and write operations of several state-of-the-art consistent data stores.

\begin{table}
  \center
    \begin{tabular}{ lccc}
    \hline
    \emph{System} & \emph{Block Size} & \emph{kRead/s} & \emph{kWrite/s} \\ \toprule
    Spanner \cite{Corbett:2012uz} & 4kB & 11 & 4 \\ 
    Spinnaker \cite{Rao11} & 4kB & 45+ & 4 \\  
    SCKV-Store \cite{Bes13} & 4kB & N/R & 4.7 \\ 
    Zookeeper \cite{Hun10} & 1kB & 87 & 21 \\ \bottomrule 
    \end{tabular}
  \caption[Performance of state machine replication systems]{Throughput (in thousands data block reads and writes per second) of consistent and fault-tolerant data stores based on state machine replication (N/R = Not Reported).}
  \label{table:smr-results}
\end{table}

Given the differences in the design and the environments where these measurements were taken, we present these values here only as supporting arguments for the possibility of using consistent data stores for storing the relevant state of SDN control applications.
Depending on the specific application this state may include, for instance, a subset of the Network Information Base (NIB).
Interestingly, these values are of the same order of magnitude of the reported values for \emph{non-consistent} updates in Onix (33k small updates per second considering 3 nodes~\cite{koponen2010}), and much higher than the reported values for their consistent data store (50 updates/second for transactions with a single update).
The Onix paper does not describe how its consistent database is implemented but, as shown by these results, its performance is far from what is being reported in the current literature.


\section{Consistent Data Planes}
\glsresetall
\label{sec:relatedWork:consistentPlane}


Recent work on SDN has explored this need for consistency at different
levels. Programming languages such as Frenetic~\cite{Foster2011} offer
consistency when composing network policies (automatically solving
inconsistencies accross network applications' decisions). Other
related line of work~\cite{reitblatt2012abstractions} proposes
abstractions to guarantee data-plane consistency during network
configuration updates. The aim of these systems is to guarantee
consistency \textit{after} the policy decision is
made. Onix~\cite{Koponen:2010:ODC:1924943.1924968} provides a
different type of consistency: one that is important \emph{before} the
policy decisions are made. Onix provides network state consistency ---
both weak and strong --- between different controller instances. The
datastore we propose is similar in that it offers strong consistency
for network (and application) state between controllers\footnote{By
  strong consistency we mean that all accesses to the datastore are
  seen by all controller instances in the same order
  (sequentially).}. Our main objective is to improve the ``performance
limitation''~\cite{Koponen:2010:ODC:1924943.1924968} of Onix's
transactional persistent database. 


Other works on SDN consistency include the Frenetic programming language~\cite{Foster2011}, Reitblatt et al. work on abstractions for network update~\cite{reitblatt2012abstractions}, and Software Transactional Networking~\cite{Canini:2013:HotSDN:STN}. 
These papers address different consistency issues, however. 
In essence, they target consistent flow rule updates on switches, dealing with overlapping policies and using atomic-like flow rule installation in SDN devices.
In other words, they take care of data-plane consistency \emph{after} the policy decisions are made by the network applications. 
Frenetic offers a high level language and a run-time system that, besides other benefits, ensures that policy rules made by concurrent network applications are not overlapped, thus avoiding network inconsistencies.
In~\cite{reitblatt2012abstractions} the authors propose two abstractions --- per-packet and per-flow consistency --- and mechanisms to guarantee that on a network update a packet (or a flow) is processed by exactly one consistent global network configuration. 
STN extends these works by proposing a transactional solution to provide consistent rule installation in a distributed setting. 
Our work targets consistency at a different level.
Our datastore ensures strong control-plane consistency for network and/or applications state, which means policy decisions are always based on a consistent state.

\LIMPA
\chapter{Distributed Controller}
\glsresetall

\section{Shared Data Store Controller Architecture}
\glsresetall
\label{sec:heimdall:architecture}
%%%%%%%%%%%%%%%%%%%%%%%%%%%%%%%%%%%%%%%%%%%%%%%%%%%%%%%%%%%%%%%%%%%%%%%%%%%%%%%%%%%%%%%%%%%%%%%%%%%%%

The proposed distributed control architecture is based on a set of controllers acting as clients of the fault-tolerant replicated key-value data store, reading and updating the required state as the control application demands, maintaining thus only soft state locally.
There are two main concerns around this design: (i) how to avoid the storage being a single point of failure and (ii) how to avoid making the storage a bottleneck for the system.
In the previous section we showed that state-of-the-art state machine replication can be used to build a data store that solves both these concerns.

Fig. \ref{fig:architecture} shows the architecture of our shared data store distributed controller.
The architecture comprises a set of SDN controllers connected to the switches in the network.
All decisions of the control plane applications running on the distributed controller are based on OpenFlow events triggered by the switches and the consistent network state the controllers share on the data store.
The fact that we have a consistent data store makes the interaction between controllers as simple as reading and writing on the shared data store: there is no need for code that deals with conflict resolution or the complexities due to possible corner cases arising from weak consistency.

By design, the SMR-based data store is replicated and fault-tolerant (as in all designs discussed in the previous section), being up and running as long as a majority of replicas is alive~\cite{Lam98}.
In other words, $2f+1$ replicas are needed to tolerate $f$ simultaneous crashes.
Thus, besides offering strong consistency, this architecture leads to a completely fault-tolerant control plane.
Furthermore, in this design the controllers keep only soft state locally, which can be easily reconstructed after a crash.
The switches tolerate controller crashes using the master-slave configuration introduced in OpenFlow 1.2\,\cite{ONF2011}, which allows each switch to report events to up to $f+1$ controllers (being $f$ an upper bound on the number of faults tolerated), with a single one being master for each particular switch.
The master is constantly being monitored by the remaining $f$ controllers, which can takeover its role in case of a crash.

Interestingly, our architecture could also be used in SDN deployments were a distributed controller is not necessary, to implement fault tolerance for centralized controllers.
In this case the fault-tolerant data store can be used to store the pertinent controller state, making it extremely simple to recover from its crash.
In this case, the applications deployed on the primary controller manage the network while a set of $f$ backup controllers keep monitoring this primary, just as in the distributed controller design.
If the primary fails, one of the backups -- say, the one with the highest IP address -- takes the role of primary and uses the data store to continue controlling the network.

Our distributed controller architecture covers the two most complex fault domains in an SDN, as introduced in~\cite{kim2012}.
It has the potential to tolerate faults in the controller (if the controller itself or associated machinery fails) by having the state stored in the fault-tolerant data store.
It can also deal with faults in the control plane (the connection controller-switch) by having each switch connected to several controllers (which is ongoing work).
The third SDN fault domain --- the data plane --- is orthogonal to this work since it depends on the topology of the network and how control applications react to faults.
This problem is being addressed in other recent efforts~\cite{kim2012,Reitblatt2013}.

\begin{figure}
\centering
\includegraphics[scale=0.6]{./pic/heimdall/multicontroller.pdf}
%add desired spacing between images, e. g. ~, \quad, \qquad etc.
%(or a blank line to force the subfigure onto a new line) 
\caption[Heimdall Architecture]{The shared data store controller
  architecture with each switch sending OpenFlow messages to two
  controllers. The controllers coordinate their actions using a
  logically centralized data store, implemented as a set of
  synchronized replicas. }
\label{fig:architecture} 
\end{figure}

\section{Floodlight} 
\glsresetall
\label{sec:heimdall:floodlight}
%%%%%%%%%%%%%%%%%%%%%%%%%%%%%%%%%%%%%%%%%%%%%%%%%%%%%%%%%%%%%%%%%%%%%%%%%%%%%%%%%%%


\section{Data Store}
\glsresetall
\label{sec:heimdall:dataStore}
%Features  design, etc., 

FIXME : Alysson - será que posso justificar que a data store
performance nao é importante quando comparada com o middleware? 

\subsection{Smart}
we had to change the library used of netty from netty-3.1.1.GA.jar to
netty-3.2.6.Final.jar because it conflicted with floodlight. We are in
the dark here. Do not known if this will cause problems... 

\subsection{Design}

\subsubsection{Map interface} 
Actually motivated by the LearningSwitch application. The rest interface exposed the learning switch database (hash table ) directly. We implemented a class to Delegate the map interface methods to our KeyValueTable. We don't actually use it inside the applications we modified because we have some preference for static typing (which for arguably legitimate reasons does not appear in the Map interface). Nevertheless we could use it. putAll and containsValue are not implemented. No special reason just lazinesss.  

LinkedHashMap should be used for maintaining consistent ordering
across replicas. 

\subsection{Cross References tables}
\label{sec.datastore.cross.references}
\subsection{Versions}
It can be really painfull to use replace if equal based on byte
representation. The serialization process may inocently output two
different outputs for values that are equal (as specified by their
equals method contract).  This can happen for something as simple as
using differen List implementations or some more obscure reason (e.g.,
when creating a deep copy object the method used to copy a list may
cause some state such as the total size of an arraylist different from
the original, if that value ends up affecting the output of the
serialization process we end up with different values). 
Example: Lists.newArrayList(existentList) from guava turned out to
have a different byte representation from  simply creating a new array
list and iterating over the existentList and insert one by one. We
kind of lost 

So all in all
the usage of replace based on byte comparison is not
advocated. Instead timestamps end up benefiting the user by being more
space efficient in the message exchanged (with a possibly
insignificant cost for carrying the timestamp value); and by being
easier to work with. 


TODO measure the cost of a timestamp? 
\subsection{Cache}
\subsubsection{Cache? What about Consistentcy}
Fine grained control. 
Monotolically increase. problems caused with clocks. Use aphry blog
post to talk about this kind of problems. 

\subsubsection{Implementation}
So for the cache implementation we actually favored composition over
composition over inheritance \ref{joshua98}  . The
\texttt{CachedKeyValueTable} can be composed of any kind of
\texttt{IKeyValueTable} and we use forwarding (i.e., the delegation
pattern) to implement all methods. However, the cache layer implement
on top of the key value table performs all the cache logic when
retrieving and updating values. So we actually update the cache values
when updating the data store. Given that we are building on top of the
version interface this complicates things, since we would actually
like need to know which is logical version the value will be
associated in the data store. In some cases we can exploit the
method sematics to achieve this. For example with \texttt{put} we get,
as a return value, the previously present value. So, we can actually
keep the newly inserted value in the cache together with its accurate
version (previous version plus one)\footnote{This actually requires
  the data store to expose and compromise its version update
  algorithm. Changing it afterwards will off course, break the client
  implementation. } As for other methods as \texttt{insert} where the
return value only indicates the success of the operation, we can not
actually know which version the newly inserted value will have in the
data store. We are faced with two choices: to leave it out of the
cache or keep it with no version associated to it.  

We choose to keep it out of the cache. Otherwise we add more
complexity to the client code which now must deal with one more outcome
when retrieving data from a cached data store. In general we do not
explicit compromise to anything in the current implemented
interface. The client can not be sure of what will get inserted in the
cache or not. This leave space open for future optimizations, or
changes  without impacting the client code correctness. But overall
the implementation on which this document is based actually updates
the datastore value on the following methods. Every retrieval method
such as get , getByReference, 
updates or writes such as : put,  replace(key, knownVersion,newValue),
putIfAbsent,    
but those don't : replace (key, existentValue, newValue), remove,
insert. 

So although it may be argued that the Cache is a implementation detail
that should not be explicit (visible ) at the interface level we
actually disagree in our case. Since the cache provides functional
behaviour: fine-graine control, explicit control over consistency
level of the data obtained from the data store. So the client, for
correctness or optimizations reasons should be able to control whether
the values obtained from the data store must be obtained with strong
consistency semantics or not. Furthermore this should it should be
able to make this choice dynamically in each call.  So our API must be
designed with all this considerations in mind. 

Cache are a common source  source of memory leaks \cite{joshua98}  in java
applications since it is easy to forget values kept in the cache long
after they are not need anymore.  

It is also relevant that cross referenced values (section
\ref{sec.datastore.cross.references} ) can be obtained in cache. So we
also a method \texttt{getValueByReference(key, ts)} to this end. 

\texttt{getCached(key)}  returns a cached value or nothing. Definitely
does not goes to the data store. 

\texttt{get(key, delta} on the other hand obtaines the value from cache
if the value is present and if the  difference between the current time and the time the value was
added  is less or equal than the specified \texttt{delta}. Otherwise
the data store is used. 

What about null values? The cache does not allows null values to be
set as well as all other data store utilities.  FIXME actually null
would be cool to known that the values are not present in the data
store. 

It works on cross references as well (getValueByReference(K key , long
ts) ). 


With cache and cross references values you don't get read your writes
consistency. That is because values are written through other tables
interface. So we have no way of updating the cache . 

\subsection{Columns}
The basic idea is to minimize the size of messages by being able to
selectiveley access the sub-parts of data store values. These
sub-parts, the attributes of an object are usually a much smaller part
than the whole, and in our experience with floodlight applications,
there are a few required components  FIXME : lower, higher  (specify
how much of the object space is used at maximum and minimum in the
flow treatment)

So, in such case it is obvious that we benefit from acessing only
those componenents. We compose a prototype to analze the benefits
that smaller messages will bring to the applications. 


It becomes complicate when we deal with cross references
tables. Because the key value store and colum based store are actually
two separated entities. 

%%% Local Variables: 
%%% mode: latex
%%% TeX-master: "../PEI"
%%% End: 
fe
\LIMPA
\setlength{\epigraphwidth}{.95\textwidth}
\chapter{Feasibility Study}
%%%% quote 

%%%%

\glsresetall


%!TEX root = ../PEI.tex
\label{sec:feasibility:apps}
\glsresetall

To evaluate the feasibility of our distributed controller design we implemented a prototype of the previously described controller architecture by integrating applications from the Floodlight controller~\footnote{\url{http://www.projectfloodlight.org/floodlight/}} with a data store built using a state-of-the-art state machine replication library, BFT-SMaRt~\cite{smart-tr,bft-smart:2011:High-perfomance}.
We considered three SDN applications provided with Floodlight: \emph{Learning Switch}  (a common layer 2 switch), \emph{Load   Balancer} (a round-robin load balancer) and \emph{Device Manager} (an application that tracks devices as they move around a network).
In this chapter we describe how they work and how we modify them. 
Then we expose particular workloads that they apply to the data store, as reaction of data plane events (e.g., a new flow), that are later used in our analysis of the data store performance.

\todo{move around a network? Queres mesmo falar sobre isso?}

%The applications were slightly modified but  we made an effort to avoid behavioral changes to the applications. The main change was shifting state from the controller's (volatile) memory to the data store efficiently (i.e., always trying to minimize communication). To the best of our knowledge the exposed services of the applications we changed are virtually indistinguishable from their predecessors.

%The objective of the experiments covered in this chapter  is to analyze the workloads generated by these applications to thereafter measure the performance of the data store when subject to such realistic demand caused by real applications.


Workloads are a simple trace (or log) of data store requests. They are a defined as a product of a data plane event, controller application and system global state (controller and data store).
Fig. \ref{fig:feasibility:workloads} exemplifies this definition: to the left we see a data plane event, triggered from the switch to the controller that, in turn, exchanges a specific sequence of messages with the data store (at the right) required to answer the event.
We will show different workloads for the three applications modified, that cover all the possible data plane events that cause an interaction with the data store. We also reveal the iterative process that defined our work by showing the incremental performance improvements done to applications with each of the data store functionalities described in section~\ref{sec:heimdall:datastore:functionalities}. Those functionalities are   a consequence of the study of the workloads existent in the initial integration of the applications (in their original state), to the data store. We believe this iterative process is valuable and should be documented, since it helps understanding what kind of bottlenecks and patterns are far from optimal when adapting existent centralized applications to our design.  

But, more importantly we use workloads to perform our feasibility study. 
We do it in three phases. 
First, we emulated a network environment in Mininet  --- a network emulation platform that enables a virtual network, running a real kernel, switch and application code on a single machine~\cite{Handigol:2012t} ---  that consisted of a single switch and a least a pair of host devices.
Then \gls{icmp} requests (pings) were generated between pairs of host devices. 
The objective was to create \gls{of}  traffic (\texttt{packet-in} messages) from the ingress switch to the controller.
Then, for each \gls{of} request, the controller performs a variable, application-dependent number of read and write operations, of different sizes, in the data store (i.e., the \textit{workload}). 
In the controller  (the data store client) we record each data store interaction entirely (i.e., request and reply size, type of operation, etc.,)  associated with the data plane event that has caused it. 
Second, the collected workload traces were used to measure the performance of our distributed data store.
For this, we set up an environment in our cluster composed of four machines, three for the distributed data store\footnote{To tolerate the crash from a single controller ($f=1$) three replicas are needed, as explained in Section \ref{sec:heimdall:datastore:bft-smart}.} and one to simulate the data store client (the controller). 
This client will concurrently replay a simulation of the recorded workload with equal payloads (i.e., equal message type and size) as well as an additional 4 byte field representing the expected reply size of the data store response. Then a simple data store server, meant to record the throughput, will reply to the client messages. We do not use our data store implementation because it was not designed for performance. The  goal of this experiment is to evaluate the performance of the middleware that  composes the bottleneck of the data store stack (BFT-SMaRt). 
From this experience we obtain the throughput and latency of processing a concurrent workloads (or concurrent data plane events).  We perform this experience for a variable of number of concurrent clients as data store clients.In practice this clients represent different threads in one controller and/or different controllers using the shared data store. 
In the third phase, we analyze the results of the previous phases. Then we try to improve on them by using the data store functionalities referred in Section \ref{sec:heimdall:datastore:functionalities}. The process then starts from the beginning. 

Each workload (with all its composing operations)  was run 50 thousand times, measuring both latency and throughput. 
We calculated averages, minimum, maximum and standard deviations in the 90, 95 and 99th percentile. 
Unless stated otherwise the values shown in this chapter are in the 90th percentile. Appendix B \cite{support}  contains all this information in graphical and raw format (as well as the captured workload in textual information). The traces (i.e., data plane events and respective workloads) for each workload shown in this chapter are available online (\cite{support} appendix A). 
In there the we can find: a script automating the data plane events in Mininet that generate all our recorded workloads as well as the trace (or log)  of both the data plane events and data store messages characteristics (i.e.,  workloads). 
There are also instructions to use our code base to replay all the experiences. 
Each machine in the performance test had two quad-core 2.27 GHz Intel Xeon E5520 and 32 GB of RAM memory, and they were interconnected with gigabit Ethernet. 
There were running  Ubuntu 12.04.2 LTS with  Java(TM) SE Runtime Environment (build 1.7.0\_07-b10) 64 bits.
 We used Mininet 2.0 (mininet-2.0.0-113012-amd64-ovf)\footnote{Available at \url{http://mininet.org}. We had an issue with this version and corrected it following online instructions available at \url{http://goo.gl/DQ7FQF}.}. 
The Floodlight version original forked  is available online\footnote{\url{http://goo.gl/RbBXag} commit 9b361fbb3f84629b98d99adc108cddffc606521f}. Finally we used BFT-SMaRt revision \textbf{??} \footnote{\url{http://code.google.com/p/bft-smart} revision \textbf{??????}} 

\todo{check smart revision.}
\todo{In Device Manager and Load Section say that the results can vary.}

\section{Learning Switch} 
\label{sec:feasibility:ls}
\glsresetall
%%%%%%%%%%%%%%%%%%%%%%%%%%%%%%%%%%%%%%%%%%%%%%%%%%%%

\begin{figure}[ht]

  \begin{subfigure}[b]{0.5\textwidth}
                \centering
                \includegraphics[width=\textwidth]{pic/feasibility/ls-events-broadcast}
                \caption{Broadcast packet.}
                \label{fig:ls:interaction:broadcast}
        \end{subfigure}%
        ~
        \begin{subfigure}[b]{0.5\textwidth}
                \centering
                \includegraphics[width=\textwidth]{pic/feasibility/ls-events-unicast}
                \caption{Unicast packet.}
                \label{fig:ls:interaction:unicast}
        \end{subfigure}
        \caption[Learning Switch workloads]{Broadcast packets trigger a write for the source address of the respective packet. Unicast packets have to additionally read the source address port location.}
        \label{fig:ls:interaction}
\end{figure}

The Learning Switch application emulates the hardware layer 2 switch forwarding process where each switch keeps a forwarding table associating \glsplural{mac} addresses to the switch ports where they were last seen. HERE For each switch a different \emph{\texttt{MAC}-to-switch-port} table is maintained in the data store. 
Each table is populated using the source address information (i.e., \gls{mac} and switch port)  presented in every OpenFlow \texttt{packet-in} request for the purpose of maintaining the location of devices. 
After learning this location, the controller can install rules in the switches to forward packets from a source to a destination. 
Until then, the controller must instruct the switch to \emph{flood} the packet to every port, with the exception of the ingress port (where the packet came in from).
Despite being a single-reader and single writer application (each switch table is only accessed by the controller managing the switch in question), we include it here for two reasons: (i) it benefits from the fault-tolerant property of our distribution process and (ii) it is commonly used as the single-controller benchmark application in the literature~\cite{Tootoonchian:2012uia,beacon2013}. 

Fig. \ref{fig:ls:interaction}  shows the detailed interaction between the switch, controller (Learning Switch) and data store for two possible cases. 
First (figure \ref{fig:ls:interaction:broadcast}), the case for broadcast packets that require one write operation to store the switch-port  of the source address. 
Second (figure \ref{fig:ls:interaction:unicast}),   the case for unicast packets, that not only stores source information, but also reads the switch egress port for the destination address.  

In its original state this application maintained an hash table
associating a switch to another hash table relating  \gls{mac} and
\gls{vlan} to switch ports. Both this hash tables were thread-safe
(i.e., supported concurrent manipulation safely). This fits naturally
in our key-value data store client implementation. The smart reader 
will wonder why does the original application is single-reader, single
writer. This happens since the internal state is actually exposed
through the northbound \gls{api} existent.  

Also, it is critical that the switch table is limited due to resource
exhaustion. Remember that each table can potentially keep an entry for
each host present in the network. For this reason the application
limits each table to a fixed number of hosts (1K by default). Whenever
this number is reached, newly added devices replace existent
devices in the table. The eviction policy used, actually uses a
\gls{lru} eviction policy where inactives entries are the first to be
evicted. On each access to the table (get and write, even if the entry
already exists), the entry associated moves to the top of a list,
making it the most recently used. Then on eviction, we only have to replace the
bottom entry by a new one. We mimicked this behaviour in the data
store but in section \ref{sec:feasibility:lsw:cache} ahead, we explain
how we affected it. 

% way better: 
%The LRU discards the least recently used items first. For this,
%whenever an entry is accessed, it moves to the top of the
%list. Whenever and entry is added to the table, but the  capacity is
%in the limit, the last entry in the access list is removed, to give room to the new one.
But the \gls{lru} is not the only way to control the entries present
in each switch table. Learning Switch also applies timeouts (hard and
soft --- see section \ref{section:background:of})  to the flows
installed in the data plane. When they expire, a switch triggers an
\gls{of} \texttt{FLOW\_REMOVED} message (containing a source and a
destination address)to the control plane which, in
turn, deletes the associated entry from the data store and instructs
the switch to remove the reverse entry (from destination to source
address) from its table. This process can then be repeated one more
time. 

\subsection{Broadcast Packet}
Figure \ref{fig:ls:interaction:broadcast}
This workload corresponds to the operations performed in the data
store when processing broadcast packets in a \acrfull{of} 
\texttt{packet-in} request.  Table \ref{table:lsw0:broadcast} shows that for the
purpose of associating the source address of the packet to the ingress
switch-port where it was received, the Learning Switch application performs one
write operation with a request size of 113 bytes and reply size of 1
byte. 

\begin{table}[ht]
\centering 
\begin{tabular}{l c c c c}
 Operation & Type & Request & Reply \\ \toprule 
 Associate source address to ingress port & W & 113 & 1 \\ \bottomrule
\end{tabular}
\caption[Workload lsw-0-broadcast( Broadcast Packet) operations]{Workload lsw-0-broadcast( Broadcast Packet) operations and sizes (in bytes).}
\label{table:lsw0:broadcast}
\end{table}
\subsection{Unicast Packet}
Figure                \ref{fig:ls:interaction:unicast}
Workload \textbf{lsw1-1} is the result of processing an \acrfull{of} request
triggered by an unicast packet. This workload builds on the previous
one, showing an additional operation to query the switch-port location of the
destination address. Table \ref{table:lsw0:unicast} provides a summary all the data
store operations in this workload. 

\begin{table}[ht]
\centering 
\begin{tabular}{l c c c c}
 Operation & Type & Request & Reply  \\ \toprule 
 Associate source address to ingress port & W & 113 & 1\\\midrule
Read egress port for destination address & R & 36 & 77 \\\bottomrule
\end{tabular}
\caption[Workload lsw-0-unicast( Unicast Packet) operations]{Workload lsw-0-unicast( Unicast Packet) operations and sizes (in bytes).}
\label{table:lsw0:unicast}
\end{table}

\subsection{Optimizations}
The Learning Switch workloads are very simple. So there is
not much we can do to improve them. Still, we noticed that there were
some extreme overhead to the messages sizes given the content that is
actually exchanged between the data store: a MAC address (8 byte
standard); and  a switch port identifier. The reason is simple, we
were the standard Java serialization process to
transform objects into byte arrays (used in the data store), and
vice-versa. This incurs in some standard overhead required by the Java
process. If we do it manually we can improve a lot on the size of the
messages exchanged as seen in table
\ref{table:lsw1:unicast}. Considering the total size of the messages
(request $+$ reply) we have actually reduced the equivalent unicast workload (table
\ref{table:lsw0:unicast} by 72\%. The same goes for the broadcast,
workload (the first message of table \ref{table:lsw-1-unicast}. 

Figure \ref{fig:lsw:comparison}  shows the results of the performance
analysis made to the four workloads. Each workload  curve has different points
taken from tests with increasingly client numbers. Accurate values and standard
deviations, as well as measures in the 95th and 99th percentile, can
be seen online \ref{support} (appendix ?).  It was somewhat surprising
that the difference in the performance between the original versions
(prefixed by lsw-0) and the optimized size versions (prefixed by
lsw-1) is unnoticeable (in some cases lsw-1 is worst than lsw-0 due to
the statistical variance). It can be that the network packet exchanged by
client and the data store (or between the data store replicas) is
actually not affected (in size) by this change. However, we will soon
verify that size optimizations are bearably unnoticeable in all our examples, except
when differing in order of magnitude. 

But we can see a significant difference between unicast and broadcast
workloads due to the fact that the former requires one more message
than the former. For the broadcast workload we can support 20kFlows/s
with a 3 ms latency. For the broadcast workload we have 12kFlows/s
with the same 3ms latency. 

The natural conclusion we can take, is to think that if we merge the
two messages that compose the broadcast workload into one (by using
Micro Components - see section \ref{sec:heimdall:datastore:mc}) we
should obtain performance results equivalent to the broadcast
workload. This is true, considering that if we add up the
lsw-1-unicast message sizes we get a very similar workload to
lsw-1-broadcast. However we do not do this, since that with cache, we
can potentially obtain much better results without requiring this. 

\begin{table}[ht]
\centering 
\begin{tabular}{l c c c c}
 Operation & Type & Request & Reply \\ \toprule 
Associate source address to ingress port & W & 29 & 1\\\midrule
Read egress port for destination address & R & 27 & 6 \\\bottomrule
\end{tabular}
\caption[Workload lsw-1-unicast( Unicast Packet) operations]{Workload lsw-1-unicast( Unicast Packet) operations and sizes (in bytes).}
\label{table:lsw1:unicast}
\end{table}

\begin{figure}[ht]
\centering
\includegraphics[scale=0.5]{../data/reportGenerator/lsw-0-broadcastlsw-0-unicastlsw-1-broadcastlsw-1-unicasttxLatCmp.pdf}
\caption[Learning Switch workloads performance comparison]{Learning
  Switch workloads performance comparison (90th percentile). }
\label{fig:lsw:comparison}
\end{figure}


\subsection{Cache}
\label{sec.learning.switch.lru.cache} 
\label{sec:feasibility:ls:cache}

Given that Learning Switch is a single reader, single writer
application, we can introduce caching mechanisms without impairing the
consistency semantics. To all effects, the Learning Switch can contain
an exact copy of everything that is present in the data store at every time.

As so we can completely avoid the data store as long as we have
entries in cache.
First we can avoid re-writing the source address to source port association when we already now it.
in the original Learning Switch this re-write is not costly
($\Omega(1)$) and has the functional impact of refreshing the entry
timestamp such that the least recently used table can keep up
consistently with the last active host and delete the inactive ones
(that may have moved or disconnect). But with the data store this is
much more expensive. 
Now, the active host
actually gets forgotten somewhere in time as newly (unknown) entries
are added to the data store, we expect this to be ok since the host,
being active, will benefit in latency a lot before actually being
erased from the data store due to the newly added hosts. 
When avoiding this write in the cache implementation we must actually be sure that we only
avoid to write to the data store when the association is known in
cache and it is actually correct (the ingress port is the same from
the packet being processed). 
The second avoidance is the read operation that queries for the egress
port of the current processed packet. We do not actually need to read
from the data store if the entry is present in the cache. First the
data is not modified by any other controller since we are the only
ones which manipulate our switch tables. 

With this improvement we no longer have to read values from the
database. We do not need since when we update the data store  (with
writes) we also update the cache. So if it isn't on the cache it is not in the
data store. This means that we can avoid our cache timed interface
(see section \ref{sec:heimdall:cache}) and just read from the data
store every time. Off course this means that the cache must be able to
handle the same state size that the data store (which may be
infeasible). 

The reader may think of an exception where we find stale data:
when a host moves from a switch to another, the tables from the first
switch will have  incorrect data and devices will be unreachable from
that switch from some time. But this also happens with the centralized
version also. This is why rules installed in the switches must have a
idle and hard timeout set. When one of the timeouts expire the switch
triggers an \texttt{FLOW\_REMOVED} message to the controller, that in
turns deletes the respective information in the data store.  This kind
of problems reside on the data plane consistency side. Not on the
control plane. 




Cache can only improve on the overall analysis of the overall system
(controller and data store) since the cache logic actually resides on the
controller and not the data store.

\todo{Finish up with reference or explanation of why we did not do it}

%We don't improve on the workload.  
%We don't actually improve on the micro-benchmarks tested measures
%shown throughout this chapters. We do not improve simply because with
%cache we do not avoid or improve (by size reduction) any of the data
%store interactions present in table \ref{table:work:lsw1-1} (that
%shows the latest learning switch workload).  With cache we will only
%improve on the long run, since we can now avoid the two type of
%requests present in that table.


\section{Load Balancer}
\label{sec:feasibility:lb}
\glsresetall
%%%%%%%%%%%%%%%%%%%%%%%%%%%%%%%%%%%%%%%%%%%%%%%%%%%%%%%%%%%%%%%%%%%%%%%%%%%%%%%%%%%%%%%%%%%

\subsection{Load Balancer main idea} 
The Load Balancer application employs a round-robin algorithm to distribute the
requests addressed to a \gls{vip} . 
In order to understand its behaviour we will begin by the data model currently used. Figure
\ref{fig:lb-model} shows the three different entities used in the Load
Balancer. The \gls{vip} entity represents a virtual endpoint with a specified \gls{ip}, port and
protocol address. Each \gls{vip} can be associated with one or more pools of 
servers. Given that the distribution algorithm is round-robin, each pool
has a current assigned server (\texttt{current-member} attribute in the figure). Finally, the third entity --- Member
--- represents a real server. 
Each of those entities, corresponds
to a different  table  in the data store, indexed by the entity
key attributes represented in the figure (in bold). Moreover, a fourth
table, named \texttt{vip-ip-to-id} is required to associate \gls{ip} addresses to \gls{vip} resources. 



\begin{figure}[ht]
\TopFloatBoxes
\begin{floatrow}
\ffigbox{
\includegraphics[scale=0.6]{./pic/feasibility/lb-model.pdf}
}{\caption{\small Simplified Load Balancer entities data model. The data
store contains a table for each entity, indexed by their keys (represent as bold attributes). }
\label{fig:lb-model}}


\capbtabbox{
\small
\begin{tabular}{cccc}
  Name & Key & Value & \\ \toprule
vips  & vip-id  & vip   \\\midrule
pools & pool-id &  pool \\\midrule
members & member-id  & member    \\\midrule
vip-ip-to-id &  ip & vip-id   \\\midrule
\end{tabular}
}{\caption[Load Balancer key-value tables]{Load Balancer key-value tables.}
\label{tablle:lb:indexes}}
\end{floatrow}
\end{figure}

In light of this data model, the load balancer logic requires the following
operations from the data store: (i) check if the source address is
associated with a VIP resource; (ii) if so, read the VIP, Pool and
Member information required to install flows in the switch and (iii)
update the pool \texttt{current-member} attribute. This description
corresponds to the case where \gls{of} \texttt{packet-in} requests are indeed addressed at a \gls{vip}
resource. The respective workload which, is the heavier in
the Load Balancer application, is represented in Fig. \ref{fig:lb:interaction:ip2Vip}. Alternatively, Fig. 
\ref{fig:lb:interaction:arp2VIp}  considers the special case of ARP requests questioning the hardware
address of a \gls{vip} \texttt{IP}. In the following sections we
expand on the details and improvements related to both those
workloads. 

\begin{figure}
  \centering
  \begin{subfigure}[b]{0.5\textwidth}
                \centering
                \includegraphics[width=\textwidth]{pic/feasibility/lb-events-broadcast}
                \caption{ARP packet address at a VIP.}
                \label{fig:lb:interaction:arp2Vip}
        \end{subfigure}%
        ~
        \begin{subfigure}[b]{0.5\textwidth}
                \centering
                \includegraphics[width=\textwidth]{pic/feasibility/lb-events-unicast}
                \caption{IP packets addressed at a VIP. }
                \label{fig:lb:interaction:ip2Vip}
        \end{subfigure}
        \caption[Load Balancer workload events]{A \texttt{\gls{arp}} request message addressed at a VIP \gls{ip} that results in a direct \gls{arp} reply. On the left a normal \gls{ip} packet addressed at VIP should be resolved (who is responsible) and replied by installing the appropriate rules}  
        \label{fig:lb:interaction}
\end{figure}

\subsection{Packets to a VIP}
When the Load Balancer receives a data packet addressed
at a \gls{vip}, it triggers the operations seen in table \ref{table:lbw-0-ip-to-vip}. 
The first two operations fetch a \gls{vip} resource associated with the
destination \gls{ip} address of the packet. The first fetches the
\gls{vip} unique identifier associated with the destination
\gls{ip}. If it succeeds, the reply is different from 0 and it can
proceed to the second operation where the \gls{vip} entity is fetched
from the data store. 
Following it, the third operation fetches the chosen pool for the returned  \gls{vip} \footnote{The current implementation of this
application always chooses the first existent pool. This behaviour
exists because the original developers anticipated a more robust Load Balancer application
where different pools can be associated with a \gls{vip} to enhance
the balancing algorithm with feedback of the application servers or
possibly the network state.}.

Afterwards it updates the fetched  pool with the newly modified
\texttt{current-member}. The forth and final operation retrieves
the address information for the selected information. 

\begin{table}[ht]
\small
\centering 
\begin{tabular}{l c c c c}
 Operation & Type & Request & Reply \\ \toprule 
Get VIP id for the destination IP & R & 104 & 8\\
Get VIP Info (pool information) & R & 29 & 509\\
Get the chosen pool & R & 30 & 369\\
Conditional replace pool after round-robin & W & 772 & 1\\
Read the chosen Member & R & 32 & 221 \\
\end{tabular}\caption[Workload lbw-0-ip-to-vip( IP packet to a VIP)
operations]{Workload lbw-0-ip-to-vip( IP packet to a VIP) operations
  and sizes (in bytes).}
\label{table:lbw-0-ip-to-vip}
\end{table}



\subsection{ARP Request}
This workload  results  from processing an ARP Request addressed at a
\gls{vip} address. The data store operations, summarized in Table
\ref{table:lbw-0-arp-request}, shows that two reads are
required. First, as previously seen,  it queries the data
store to check if the packet destination address is a \gls{vip} (1 read
needed). As it is, the controller then retrieves the \texttt{MAC} address for that
\gls{vip} server (so, another read is needed to obtain the \gls{vip} entity).

\begin{table}[ht]
\small
\centering 
\begin{tabular}{l c c c c}
Operation & Type & Request & Reply \\ \toprule 
Get VIP id for the destination IP  & R & 104 & 8\\
Get VIP info (proxy MAC address) & R & 29 & 509 \\\bottomrule
\end{tabular}\caption[Workload lbw-0-arp-request( Arp Request to a
VIP) operations]{Workload lbw-0-arp-request( Arp Request to a VIP)
 operations and sizes (in bytes).}
\label{table:lbw-0-arp-request}
\end{table}

\subsection{Optimizations}
\begin{table}[ht]
\small
\begin{tabular}{llccccc}
 Operation & Type &  \multicolumn{5}{c}{ (Request, Reply) } \\  \midrule
&  & lbw-0 & lbw-1  & lbw-2 & lbw-3 & lbw-4 \\ \toprule 
%& &   \multicolumn{5}{c}{(Request, Reply)} \\midrule 
Get VIP id of destination IP  & R & (104,8) &\multirow{2}{*}{(104,509)} &  \multirow{2}{*}{(104,513)} &\multirow{2}{*}{\textbf{(62,324)}} & \multirow{2}{*}{-}    \\\cmidrule{1-2} 
Get VIP info (pool)   & R &  (29,509) & & & &   \\ \midrule 
Get the choosen pool  & R & (30,369)  &  - & (30,373) & -   & \multirow{3}{*}[-2mm]{\textbf{(11,4)}}  \\  \cmidrule{1-2} 
Replace pool after round-robin  & W & (772,1) & -
&\textbf{(403, 1)} &  - \\ \cmidrule{1-2}  
  Read the chosen Member &  R & (32,221) & - & (32,225) & \textbf{(44,4)} & \\\bottomrule  
\end{tabular}\caption[Load Balancer IP to VIP workload operations across
diferent implementations.]{Load Balancer  lbw-\textit{X}-ip-to-vip workload
  operations and respective sizes (in bytes) across diferent
  implementations. Bolded sizes represent significant differences
  across implementations. Sizes marked with \texttt{-} are equal to
  the previous. } 
\end{table}

\begin{figure}[ht]
% \CenterFloatBoxes
%\TopFloatBoxes  
% \BottomFloatBoxes
\begin{floatrow}
\ffigbox{%
  \includegraphics[scale=0.4]{../data/reportGenerator/lbw-0-ip-to-viplbw-1-ip-to-viplbw-2-ip-to-viplbw-3-ip-to-viplbw-4-ip-to-viptxLatCmp.pdf}
}{\caption{Cenas}%
}
\capbtabbox{%
\small
  \begin{tabular}{lll} 
    Prefix &  Data store & Section\\\toprule
    lbw-0 & Simple Key-Value  & \ref{sec:}  \\
    lbw-1 & Cross References  & \ref{sec:} \\
    lbw-2 & Versioned Values & \ref{sec:} \\
    lbw-3 & Column Store & \ref{sec:} \\
    lbw-4 & Micro Components & \ref{sec:} \\ \bottomrule
    & &  \\ 
    & &  \\ 
    & &  \\ 
    & &  \\ 
    & &  \\ 
  \end{tabular}
}{%
  \caption[Name guide to Load Balancer workloads]{Name guide to Load
    Balancer workloads.}\label{table:lb-versions}
}
\end{floatrow}
\end{figure}

Table \ref{table} shows us the all the modifications we have done to
the workload triggered by a packet addressed at a \gls{vip}
address. It is somewhat similar to previous workload tables (e.g,
\ref{tables}), but this time, we show the workload according to
different implementations of the application using a variety of data
store functionalities. This way we can verify the impact of each
functionality individually. For reference we prefix the workload name
with a different name: lbw-0, lbw-1, ..., lbw-4. The functionality
used with that prefix can be seen in table
\ref{table:lb-versions}. The prefix lbw-0 refers to the simple
key-value store implementation which has already been presented in
table \ref{table:lbw-0-ip-to-vip} that summarized  the workload 
operations. Message size is now grouped in tuples. When no difference is noticed between implementations we use the \texttt{-}
symbol instead of a tuple. Significant changes are emphasized in
bold. Also notice that some operations are merged together (first
two operations on workload lbw-1 and the last three in workload
lbw-4.

For the first improvement we eliminate the double step required to
obtain a \gls{vip} in the first two messages. This can be done with
the Cross Reference functionality by configuring the data store with
the information that each value of the \texttt{vip-ip-to-id} table
(consulted in the first workload operation) is actually a key for
the \gls{vip} table. Then the data store, can  --- in one operation
--- fetch the \gls{vip} for the provided \gls{ip} address. 

Next, in workload lbw-2 we cut an operation size in half, by replacing the conditional
replace (after round-robin) in the 4th line by a similar operation
that uses a version number, provided by the data store while reading
the \gls{vip} information (notice the increase by 4 bytes in the first
read caused by adding the version number of the \gls{vip} to the
reply). Now, by logical time stamping every data store value, we avoid
the use of replace operations that requiring sending both the old
value and the new value. This not only benefits size but also
simplifies client code (see section \ref{sec:heimdall:versions}). 


We improve from workload lbw-2 to lbw-3 by adding Column support to
the data store. Then we can replace the existing reads of \gls{vip}
(first two operations) and Members by reading  them partially instead
of reading them entirely. With \gls{vip} entities we do not actually
improve a lot because the local operations that follow the read in the
data store actually require a lot of the attributes of a \gls{vip} (it
will become clear why in the next section). But with Members we
improve by a factor of 56 in the return value because we only require
reading its \gls{ip} attribute. 

Final we get to the most significant improvement by using
Micro-Components to provide a data store method that performs the
round robin operations and returns the Member \gls{ip} in only one
step. This table actually does not shows the true beneficial of this
optimization. Before it, we actually fetched and updated a pool in two
separate steps with the conditional replace. This is a potentially
bottleneck under high-peak utilization of the balanced resources
because we suspect this optimistic concurrency control mechanism will
fail a lot when different controllers receive significant concurrent
traffic addressed at the same \gls{vip}.  In the next section it will become clear why it is beneficial
to have two separate methods to fetch a \gls{vip} and round-robin. \\

\todo{Fazer um teste simple a mostrar que falhava era bem mais
  cientifico do que suspect} 

\todo{Tenho que explicar a figura de novo? Já expliquei como é que
  funciona na secção do Device Manager. Copy paste?} 

In Fig. \ref{fig:lbw:comparision} we see the performance results of
the different workloads. We can see observe the same patterns of the
previous performance analysis we done with Learning Switch (see
section \ref{section} ). Again, the size reduction improvements seem
to have little to no effect from workloads  lbw1 to lbw-3. This is
more expected since the improvements actually have a smaller  impact
in the workload then in the case of the Learning Switch. We can also
see that as before, message reduction has the greatest impact. \textbf{From
workload lbw-0 to lbw-1 we see a minor improvement from 4.5kFlows/s
with 4 ms latency to 6.1kFlows/s to 6.7 ms. }. Better than that we see
that with the final workload (lbw-4) we improve to 12kFlows/s  under
5ms latency. An improvement of more than double from the original
workload, with only a 25\% increase in latency. 


\subsection{Cache}
Caching in the Load Balancer case has effects on the consistency as
opposed to the Learning Switch case. This is because the Load Balancer
data is expected to shared and manipulated by different controllers
(unless we avoid by design and configuration). Still we explore this
possibility and defend, advocating that it may not be harmfull to do
so. 

Considering the ip-to-a-vip workload and use case. If we use cache to
maintain \gls{vip} entities locally on a time based manner. We can
benefit a lot. In the worst case the local \gls{vip} information can
be stale. Then clients may try to reach a VIP that does not exists
anymore for some reason or may have changed \gls{ip}  address to
provide another service. This is something that already can happen in
the strongly consistent version. When we fetch a \gls{vip} from the
data store it can already be invalid.  byt the time we get it
locally. But with cache this probability is greater.

\todo{We can actually solve this by using the version number in the
  round robin mechanism. Ou então ser for mesmo uma VIP vai logo à
  base de dados fazer round robin. E na cache ficas com blooms filters
  para dizer aqueles que não são vips. Falta implementar. }

On the other hand caching a \gls{vip} can be very beneficial because
the Load Balancer has a minimal impact on the controller pipeline of
one read in the data store. Even when processing a normal packet, not related to a VIP address at
all, the Load Balancer still has to find out if this is the case. This
workload, which only requires one operation (see table (First line of tables))
\ref{table:lbw-0-ip-to-vip,table:lbw-0-arp-request} but with  a 0 byte
reply) sets the minimum amount of work imposed by
the Load Balancer to the controller pipeline. 
requires this read operation. It would be great to avoid this
behaviour because we do not want to limit the pipeline to nearly 15.5kFlows/s
under 3ms latency (which is the data store performance when reading an
\gls{vip} as seen in Figure \ref{fig:}). 

As a final note, caching is actually the reason why the fetch
operation of a \gls{vip} described in the previous section actually
does not bring a significant improvement. This is not accidental but
by choice. The reason is that \gls{vip} are actually read for two
different functions: packet to a vip and arp request to a vip. So we
believe it is simpler and more efficient to actually fetch to the
cache the union of attributes required by the two different
cases. This makes sense because the two events are not independent. An
\gls{arp} request addressed at a \gls{vip} is usually followed by an
\gls{ip} packet to the same \gls{vip}.  

The time the \gls{vip} is considered valid in cache is
configurable. To be efficiently used the user should consider both the
dynamic changes frequency done to a \gls{vip} and the time under which
they take effect, as well as the probability of successive arp and ip
packets to the same \gls{vip}

%Ideally we should also avoid the normal case of IP packets not
%addressed at a VIP. For this our cache  must understand what a empty
%value means FIXME. (use containsInCache . update to insert empty in
%cache. Then see if containsInCache AND get == null you can be certain
%the value is not a VIP), completely avoiding the going to the data store.




\begin{figure}[ht]
\centering
\includegraphics[scale=0.5]{./../data/reportGenerator//lbw-3-ip-to-notviptxLat.pdf}
\caption[Minimum impact of Load Balancer in the pipeline.]{Workload
  lbw-3-ip-to-notvip shows the minimal impact the Load Balancer
  applications has on the pipeline in our best implementation.}
\end{figure}


\section{Device Manager}
\label{sec:feasibility:dm}
\glsresetall
%%%%%%%%%%%%%%%%%%%%%%%%%%%%%%%%%%%%%%%%%%%%%%%%%%%%

The Device Manager application tracks and stores host devices
information such as the switch-ports to where devices are
attached to\footnote{The original application is able to track devices as
  they move around a network, however our current implementation does
  not take this feature into consideration.}. This information ---
that is retrieved from the OpenFlow packets that the controller receives --- is crucial to
Floodlight’s Forwarding application. That is to say, that for  each new flow, the Device
Manager has the job of finding a switch-port for both the destination
and source address. Given this information, it is able to pass it to
the Forwarding application, that can later decide on the actions to
take (e.g., best route). Notice that this arrangement, excludes the
Learning Switch as the  forwarding application in action. 

Regarding the data store usage, Device Manager requires three
data store tables listed in table \ref{table:dm:indexes}.  The first
table, \texttt{devices} keeps track of known devices created by the
application. A second table named \texttt{macs}  indexes those same devices by their
\gls{mac} and \gls{vlan}  pair.  Finally, a third table named
\texttt{ips} maintains an index from an \gls{ip} address to one or
more devices.

\begin{table}
\small
\begin{tabular}{cccc}
Name & Key & Value & \\ \toprule
devices & device-id &  device \\\midrule
macs & (MAC,VLAN)  & device-id   \\\midrule
ips  & IP & device-id list \\\midrule
\end{tabular}
\caption[Device Manager key-value tables]{Device Manager key-value tables.}
\label{table:dm:indexes}
\end{table}

We will analyze and improve on  two distinct workloads for this application differing in
wether the application already knows the source device information (figure \ref{fig:dm:interaction:known})
or not ( \ref{fig:dm:interaction:unknown}). 

In the former case, the
application mainly reads information from the data store in order to
obtain location information. As for the latter case, the
application must create the device information and updates all the
existent tables. Therefore, this workload generates more traffic between
the controller and data store. 


\begin{figure}
  \centering
  \begin{subfigure}[b]{0.5\textwidth}
                \centering
                \includegraphics[width=\textwidth]{pic/feasibility/dm-unknown}
                \caption{Packet from an unknown device.}
                \label{fig:dm:interaction:unknown}
        \end{subfigure}%
        ~
        \begin{subfigure}[b]{0.5\textwidth}
                \centering
                \includegraphics[width=\textwidth]{pic/feasibility/ls-events-unicast}
                \caption{Packet from a known device.}
                \label{fig:dm:interaction:known}
        \end{subfigure}
        \caption[Device Manager workload events]{Workloads for this application heavily depend on the state of the data store. Unknown devices trigger several operations to the creation of these, while known devices only require an update of their "last seen" timestamp. No matter the case, the source and destination devices are retrieved if they exist.}
        \label{fig:dm:interaction}
\end{figure}

\subsection{Known Devices}

\begin{table}[ht]
\small
\centering 
\begin{tabular}{l c c c c}
Operation & Type & Request & Reply \\ \toprule 
Read the source device key & R & 408 & 8\\
Read the source device & R & 26 & 1444\\
Update "last seen" timestamp & W & 2942 & 0\\
Read the destination device key & R & 408 & 8\\
Read the destination device & R & 26 & 1369 \\
\end{tabular}
\caption[Workload dm-0-known (Known Devices) operations]{Workload
  dm-0-known (Known Devices) operations and sizes (in bytes).}
\label{table:ops:dm-0-known}
\end{table}

When devices are known to the application, a \texttt{packet-in} request
triggers the operations seen in table \ref{table:ops:dm-0-known}. The
first two operations read source and destination device
information in order to make their switch-ports available to the
Forwarding process. Additionally, the second operation (a write), 
updates the ``last seen'' timestamp of the source device.

\subsection{Unknown Source}
\small
\begin{table}[ht]
\centering 
\begin{tabular}{l c c c c}
Operation & Type & Request & Reply \\ \toprule 
1) Read the source device key & R & 408 & 0\\
2) Get and increment the device id counter & W & 21 & 4\\
3) Put new device in device table & W & 1395 & 1\\
4) Put new device in \texttt{(MAC,VLAN)} table & W & 416 & 0\\
5) Get devices with source IP & R & 386 & 0\\
6 ) Update devices with source IP & W & 517 & 0\\
7) Read the destination device key & R & 408 & 8\\
8) Read the destination device & R & 26 & 1378 \\\bottomrule
\end{tabular}
\caption[Workload dm-0-unknown( ARP from Unknown Source)
operations]{Workload dm-0-unknown( ARP from Unknown Source) operations
  and sizes (in bytes).}
\label{table:ops:dm-0-unknown}
\end{table}


This workload is triggered in the specific case in which  the source device
is unknown and the \gls{of} message carries an \gls{arp} reply 
packet. Seing that both these  conditions are true, the application
proceeds  with 8 data store operations, described in table
\ref{table:ops:dm-0-unknown}. Their intention is to create device
information and update the three tables described  in the beginning
of this section.  

The first operation reads the  source device key. Being
that it is not known, this operation fails (notice in the table, that
the reply has a size  of zero bytes). As a result the application
proceeds with the creation of a device. For this, the
following write (second operation) atomically retrieves
and increments a device unique \texttt{id} counter. Afterwards, the third and fourth  operation
update, with the newly created device, the device and MAC/VLAN
tables respectively. Likewise, the fifth and sixth operations update
the \gls{ip} index table. Given that this index links an \gls{ip} to
several devices we are forced to first collect the set of devices in
order to update it. This \emph{read-modify} operation can
fail in case of concurrent updates. Under that case, both operations
would be repeated until they succeed. At this point, the Device Manager
is done with the creation of the device and can, finally, move to the
last two operations to fetch the destination device information. 

\subsection{Optimizations}
\begin{figure}
  \centering
  \begin{subfigure}[b]{0.5\textwidth}
                \centering
                \includegraphics[width=\textwidth]{../data/reportGenerator/dm-0-unknowndm-1-unknowndm-2-unknowndm-3-unknowndm-4-unknowntxLatCmp.pdf}
                \caption{}
                \label{fig:}
        \end{subfigure}%
        ~
        \begin{subfigure}[b]{0.5\textwidth}
                \centering
                \includegraphics[width=\textwidth]{../data/reportGenerator/dm-0-knowndm-1-knowndm-2-knowndm-3-knowndm-4-knowntxLatCmp.pdf}
                \caption{}
                \label{}
        \end{subfigure}
        \caption[Device Manager performance analysis]{}
        \label{fig:dm:performance}
\end{figure}

\begin{table}
\small
\begin{tabular}{lll} 
    Prefix &  Data store & Section\\\toprule
    dm-0 & Simple Key-Value  & \ref{sec:heimdall:datastore:kv}  \\
    dm-1 & Cross References  & \ref{sec:heimdall:datastore:cr} \\
    dm-2 & Versioned Values & \ref{sec:heimdall:datastore:vr} \\
    dm-3 & Column Store & \ref{sec:heimdall:datastore:cr} \\
    dm-4 & Micro Components & \ref{sec:heimdall:datastore:mc} \\ 
  \end{tabular}
  \caption[Name guide to Device Manager workloads]{Name guide to
    Device Manager workloads.}
  \label{table:names:dm}
\end{table}

\begin{table}[ht]
\small
\centering
\begin{threeparttable}
\begin{tabular}{ll ccccc}
 Operation & Type &  \multicolumn{5}{c}{ (Request, Reply) } \\  \midrule
&  & dmw-0 & dmw-1  & dmw-2 & dmw-3 & dmw-4 \\ \toprule 
Get source key & R &(408, 8) & \multirow{2}{*}{(408,1274)} &
\multirow{2}{*}{(408,1278)} & \multirow{2}{*}{(486,1261)} &
\multirow{2}{*}{(28,1414)} \tnote{a} \\ \cmidrule{1-2}
Get source device & R & (26,1444) & & & & \\ \midrule
Update timestamp & W & (2942,0) & (2602,0) & \textbf{(1316,1)} & (667,1) & 
(36,0) \\ \midrule
Get destination key & R & (408,8) & \multirow{2}{*}[-1mm]{(408,1199)} &
\multirow{2}{*}[-1mm]{(408,1203)} & \multirow{2}{*}[-1mm]{(416,474)} &
\multirow{2}{*}[-1mm]{N/A} \\ \cmidrule{1-2}
Get destination device & R & (26,1369) &  &
 & & \\\bottomrule
\end{tabular}
\caption[Workload dm-0-known( Known Devices) operations]{Workload
  dm-0-known( Known Devices) operations and sizes (in bytes).}
\begin{tablenotes}
\item [a)] This operation also fetches the destination device.
\end{tablenotes}
\end{threeparttable}
\end{table}

%TODO - do not use put new device in MAC,VLAN table. This is
%confusing. 

\begin{table}[ht]
\small
\centering 
\begin{threeparttable}
\begin{tabular}{ll ccccc}
 Operation & Type &  \multicolumn{5}{c}{ (Request, Reply) } \\  \midrule
&  & dmw-0 & dmw-1  & dmw-2 & dmw-3 & dmw-4 \\ \toprule 
Read source key & R & (408,0) & - & - & (486,0) & (28,201)\tnote{a}\\
Increment counter & W & (21,4) & -  & - & - & \multirow{5}{*}{(476,8)} \\
Update device table & W & (1395,1) & (1225,1)\tnote{b}  & - &
(1183,1) & \\
Update MAC  table & W & (416,0) & - & - & -
& \\
Get from IP index & R & (386,0) & - & - & - & \\
Update IP index  & W & (517,0) & - & - & - & \\
Get destination key & R & (408,8) &
\multirow{2}{*}{(408,1208)}\tnote{b} & \multirow{2}{*}{(408,1212)} &
\multirow{2}{*}{(416,474)} & \multirow{2}{*}{N/A}  \\ 
Get destination device & R & (26,1378)  &  & & \\\bottomrule
\end{tabular}
\caption[Workload dm-0-unknown( ARP from Unknown Source)
operations]{Workload dm-0-unknown( ARP from Unknown Source) operations
  and sizes (in bytes).}
\begin{tablenotes}
\item [a)] This operation also fetches the destination device.
\item [b)] Differences in sizes caused by a SERIALIZATION improvement 
\end{tablenotes}
\end{threeparttable}
\end{table}


First we  replace the two step operation required to fetch a device by
using Cross References tables (dm-1). Just as we have done in
analogous Load Balancer workload.  Next, we substitute the replace
operation in dm-1 by a version based replace. 

Columns (dm-3) we put the device. We  nearly do not improve  while
fetching the source device since the local logic requires reading
almost all the device attributes, but at least we do not incur in any
overhead associated with the column serialization logic. On the other
end the update timestamp operation has half the request size. We do
not improve more because the update timestamp updates a specific
element inside an array, so we actually have to replace the all array
since our update is limited. As for the destination device we can
reduce it nearly by a factor of three. 

Finally with micro components we can improve a lot. First we can
merge the source and destination device in only one operation (which
could also be done with transactions also). Following we can also
create a micro component to update the timestamp in the data
store. For this we only need to send the device key and new
timestamp which greatly improves the operation size.  

Also with workload dm-4-\ref{table:} we can improve by replacing
operations 2 to 6 by only operation that crates a new device. Again,
as in the known-devices workload we can fetch the source and
destination device information simultaneously. 

We can merge all operations (reading devices and creating them)
because we need to consult a local service (the topology manager) in
the controller before creating devices.  This is because we actually
have not shown other workloads for the Device Manager application
(e.g., when updating new \gls{ip} addresses). 



\subsection{Cache}
With cache we fetch known devices to the cache. Then in the known
devices workload case when we update the timestamp with the local version
number, the operation will only succeed if we have the most recent
device version.  \textbf{We expect this replace operation to succeed a lot of
time since we do not anticipate scenarios where different controllers
manipulate the same devices. But they can happen}. In practice this is
no different as the previous situation where the replace operation
could also fail (without cache). \textbf{If the devices are connected to different openflow
islands simultaneously than this is a bad idea since we will actually
have to perform one more request that the normal workload
pattern. (try to updated - fails, retrieve new , update) . Off course
this could be mitigated by having the update attempt to return the
currently present value timestamp}


%%% Local Variables: 
%%% mode: latex
%%% TeX-master: "../PEI"
%%% End: 

 
\LIMPA

\chapter{Conclusions}
\glsresetall
% sumario de trabalho realizado

% comentario critico
% possibility of future work 
§
\section{Conclusions}
\todo{Por escrever...}
\subsection{Frow Ewsdn}
In this thesis we have proposed a distributed, highly-available, strongly consistent controller for SDNs.
The central element of the architecture is a fault-tolerant data store that guarantees acceptable performance.
We have studied the feasibility of this distributed controller by analyzing  the workloads generated by representative SDN applications and demonstrating that the data store is capable of handling these workloads, not becoming a system bottleneck.

The introduction of a fault-tolerant, consistent data store in the architecture of a distributed SDN controller has a cost.
Adding fault tolerance increases the robustness of the system, while strong consistency facilitates application design, but the fact is that these mechanisms affect system performance.
First, the overall throughput will decrease to the least common denominator, which will in most settings be the data store.
Second, the total latency will increase as the response time for a data path request now has to include i) the latency to send a request to the data store; ii) the time to process the request; and iii) the latency to reply back to the controller.
Starting by assuming the inevitability of this cost, our objective in this thesis is to show that, for some network applications at least, the cost may be bearable and the overall performance of the system remain acceptable.

The drawback of a strongly consistent, fault-tolerant approach for an SDN platform is the increase in latency, which limits responsiveness; and the decrease in throughput that hinders scalability.
Even assuming these negative consequences, an important conclusion of this study is that it is possible to achieve those goals while maintaining the performance penalty at an acceptable level.

As the number of SDN production networks increase the need for dependability becomes essential. The key takeover of this work is that dependability mechanisms have their cost, and it is therefore an interesting challenge for the SDN community to integrate these mechanisms into scalable control platforms. 

\subsection{Things we found out}
\begin{itemize}
\item Scalability, reactive model, real-world-applications: 
  \begin{itemize}
  \item The partition of the data plane across different controllers should enable scalabality. However, in our experience each event triggered by the switches triggers at least a write (in most cases) to the data store. This may hinder scalability of the control plane. 
  \item It can be that our architecture is not a best-fit for the reactive model. 
  \item However we did not take an approach of modifying  the semantic of the applications! It can be that modifed applications can behave better. (Device Manager and timestamp).  
  \end{itemize}
\end{itemize}

\subsection{Limitations}
\begin{itemize}
\item Our analysis is not thorough  because it does not accounts for the other applications effects (e.g., device manager may consult the topology manager); 
% \item It is important to clarify from the outset that the feasibility study analysis is arguably simplistic since it is focused in on data plane plane events in isolation . Considered in isolation, 
% other applications behaviour can impact on the workloads  by communicating with the application API.
% \item Networks are self-healing by nature. The use of a distributed control plane with solid (consistent)  seems to benefit optimiality and simplicity, but does not seems to solve much of the problems present in the data plane. 
\item We do not have a real data plane workload. What kind of events are more frequent? It can have a deep impact;
\item It is only a micro-benchmark, the controller performance must be taken into consideration; 
\item We need Query Languages. Reading entire tables is proibitive as we personal have seen in the Topology manager application. 
%Read entire tables, a common behaviour observed has not practical solution.  This will off course result in a giant overhead to the data store. Naturally this can be solved by introducing stale data with a cache, which may not be harmfull if the applications are not in a critical path to the network control goals (e.g., systems who display information). However we have seen this kind of code in Floodlight applications. In fact, there was one application that was left out of this work in a early stage, since after we adapt it to our data store and analyzed its behaviour we found that on each topological change to the network, it required reading all the data store state to behave 
\end{itemize}


% \begin{itemize}
% \item The application behaviour found in the controller is too demaning for consistent data stores. And it can not even justify the usage of this data stores. 
% \item An example is the Device Manager which updates the last seen timestamp on every flow arriving at the controller. If this value is not careful choosen it may impose an extreme demand to the data store
% \item Another example is the Load Balancer. Even it Levin et. al suggest that a stale view of the network incurrs in beneath optimal (when compared to a consistent view) balancing performance the behaviour that we encounter in the Load Balancer application definitely limits the controller the controller scalability severely when contrasted with the single controller in multi-cores. 
% \item This issues can be addressed with two events: either the data store supports a more scalable consistency model or the applications have to be modified to be more pro-active. By pro-active we mean that hey must stop resorting to flow requests sent to the controller to maintain an updated network view. We reason that more efficient methods can be built that can be as much flexible as the existing ones and also more efficient given an controller with our characteristics. For example the Device Manager should resort to additional  events instead of flow requests to find out mobility, while the forwarding applications could install flows that expire on activity based rule space (there are works that adress the switch table problems without resorting to least recently used like cache mechanisms). This way the switch input rate imposed on the controller can be lowered possibly in ways that can be satisfied by our current data store performance. Also the Load Balancer could resort to periodically consult the switch channels condition (usage) and update the data store. By batching the entire switch domain conditions we could have a fined grained  view of the entire network state kept consistently in the data store. The controller can even choose to have different window of inconsistencies values for different subsets of the data plane according to the importance of the Load Balancer activity. 
% \end{itemize}


% \begin{itemize}
% \item So in the best case we have 200kFlows/s with a data store.  That 2 out 3 applications that are fundamental in the packet in pipeline require at least one write operation exchanged with the data store. So even with top-notch engineering techniques of caching, batching, non-blocking I/O , we are definitely  setting a hard limit on the controller scalability. In fact this drop the processing rate of state of the art centralized controllers by a factor of 5 (at  least!). 
% \item We distribute a control plane for three reasons: fault tolerante (reliability!), scalability and performance (in the controller case, this concerns the latency between switches and the controller). 


\subsection{Future Work}
\begin{itemize}
\item Add applications that we believe to be fundamental to the data center: Topology Manager and Firewall.  Study the pipeline effect of all the applications together and improve on that. 
\item The remaining architecture. 
\item Multiple consistency semantics, bessani protocol, relax on reads operations. 
\item ``Transactional Pipeline'', a single event, multiple applications, a single round trip to the data store. 
\item Explore the proactive model. It is a better fit for our architecture if scalability is an issue. forwarding requests are just too much. Network changes are easily dealt with.... 

% \item Study the differences between serializability and strong consistency, establishing when do the applications require one or another. State machine replication is not required in all aplications, use strong consistency in some cases (reads, writes). Bessani Protocol, cassandra with strong consistency scales better than SMR , etc., 
\end{itemize}


%%% Local Variables: 
%%% mode: latex
%%% TeX-master: "../PEI"
%%% End: 

\LIMPA

% Inicio apendices
\appendix

\include{./tex/apendices}




% Fim do conteudo
% ----------------------------------------------------------------------

% Glossario

\LIMPA

%
% Para actualizar o glossario, e' preciso correr o script ./fazindice
% e voltar a gerar o PDF
%
%\newacronym{sdn}{SDN}{Software Defined Network}

\newacronym{nib}{NIB}{Network Information Base}

\newacronym{wan}{WAN}{Wide Area Network}

\newacronym{nos}{NOS}{Network Operating System}

\newacronym{os}{OS}{Operating System}

\newacronym{onf}{ONF}{Open Network Foundation}

\newacronym{of}{OF}{OpenFlow}

\newacronym{fifo}{FIFO}{First In,First Out}

\newacronym{smr}{SMR}{State Machine Replication}

\newacronym{arp}{ARP}{Address Resolution Protocol}

\newacronym{ip}{IP}{Internet Protocol}

\newacronym{mac}{MAC}{Media Access Control}


%\renewcommand{\glossaryname}{Glossary}
%\setglossary{glodelim={\noexpand}}
%\setglossary{glsnumformat=ignore}
\printglossary
\addcontentsline {toc} {chapter} {Glossary}

% Bibliografia

% \LIMPA

\printbibliography[title=References,heading=bibintoc]

% Indice remissivo
\LIMPA
%\printindex
%\addcontentsline {toc} {chapter} {Table Of Contents}

\end{document}
